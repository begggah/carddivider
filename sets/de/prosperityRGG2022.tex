% Basic settings for this card set
\renewcommand{\cardcolor}{prosperity}
\renewcommand{\cardextension}{Erweiterung III}
\renewcommand{\cardextensiontitle}{Blütezeit}
\renewcommand{\seticon}{prosperity.png}

\clearpage
\newpage
\section{\cardextension \ - \cardextensiontitle \ 2. Edition (Rio Grande Games 2022)}

\begin{tikzpicture}
	\card
	\cardstrip
	\cardbanner{banner/white.png}
	\cardicon{icons/coin.png}
	\cardprice{3}
	\cardtitle{\footnotesize{Handelsroute}}
	\cardcontent{Du erhältst + 1 Kauf und +\coin[1] pro Geldmarker, der sich zum Zeitpunkt des Ausspielens auf dem Handelsrouten-Tableau befindet. Du musst eine Handkarte entsorgen, wenn du mindestens 1 Karte auf der Hand hast.

	\medskip

	In allen Spielen mit der \emph{HANDELSROUTE} (auch als Teil des \emph{SCHWARZMARKTES}) wird zu Beginn des Spiels das Handelsrouten-Tableau neben dem Vorrat bereit gelegt. Außerdem wird auf jeden Punkte-Vorratsstapel (\emph{ANWESEN}, \emph{HERZOGTUM}, \emph{PROVINZ}, ggf. \emph{KOLONIE} sowie alle kombinierten oder reinen Punktekarten unter den Königreichkarten (z.B. \emph{GÄRTEN} aus Basisspiel oder \emph{INSEL} aus Seaside)) ein Geldmarker gelegt. Sobald die erste Karte eines Stapels genommen wird (egal von welchem Spieler und egal auf welche Weise), legt ihr den entsprechenden Geldmarker auf das Handelsrouten-Tableau. Es wird kein neuer Marker auf den Vorratsstapel gelegt. Auch wird unter keinen Umständen ein Marker vom Handelsrouten-Tableau entfernt.}
\end{tikzpicture}
\hspace{-0.6cm}
\begin{tikzpicture}
	\card
	\cardstrip
	\cardbanner{banner/gold.png}
	\cardicon{icons/coin.png}
	\cardprice{3}
	\cardtitle{Lohn}
	\cardcontent{Diese Karte ist eine Geldkarte mit zusätzlichen Anweisungen. Sie hat den Wert \coin[1].

	\medskip

	Decke solange Karten von deinem Nachziehstapel auf, bis du die erste Geld- oder kombinierte Geldkarte aufdeckst. Entsorge die aufgedeckte Geldkarte oder lege sie ab. Alle anderen aufgedeckten Karten legst du ab.}
\end{tikzpicture}
\hspace{-0.6cm}
\begin{tikzpicture}
	\card
	\cardstrip
	\cardbanner{banner/blue.png}
	\cardicon{icons/coin.png}
	\cardprice{3}
	\cardtitle{Wachturm}
	\cardcontent{Diese Karte ist eine kombinierte Aktions- und Reaktionskarte. Sie kann in der Aktionsphase ausgespielt werden (Anweisung über der Trennlinie) oder als Reaktion auf die unter der Trennlinie angegebene Situation.

	\medskip

	Spielst du den \emph{WACHTURM} in deiner Aktionsphase aus, ziehst du solange Karten nach, bis du 6 Karten auf der Hand hast. Hast du bereits 6 oder mehr Handkarten, ziehst du keine Karten nach. 

	\medskip

	Wenn du den \emph{WACHTURM} auf der Hand hast und du eine Karte nimmst (in deinem eigenen Zug oder während des Zuges eines Mitspielers), darfst du ihn als Reaktion aufdecken und die genommene Karte entweder entsorgen oder auf deinen Nachziehstapel legen. Anschließend nimmst du den \emph{WACHTURM} wieder auf die Hand. Nimmst du anschließend eine oder mehrere weitere Karten (durch die gleiche oder eine andere Anweisung bzw. durch einen Kauf), kannst du den \emph{WACHTURM} erneut aufdecken – solange du ihn auf der Hand hast. Hast du den \emph{WACHTURM} in deiner nächsten Aktionsphase noch immer auf der Hand, darfst du ihn ausspielen. 

	\medskip

	Wenn ein Mitspieler gegen dich die \emph{BESESSENHEIT} (aus \emph{Alchemie}) gespielt hat und du den Extrazug ausführst, darfst du den \emph{WACHTURM} nicht aufdecken, da der Mitspieler die Karten nimmt und nicht du.}
\end{tikzpicture}
\hspace{-0.6cm}
\begin{tikzpicture}
	\card
	\cardstrip
	\cardbanner{banner/white.png}
	\cardicon{icons/coin.png}
	\cardprice{4}
	\cardtitle{Arbeiterdorf}
	\cardcontent{Du \emph{musst} eine Karte ziehen, \emph{darfst} 2 weitere Aktionen ausführen und in der Kaufphase einen zusätzlichen Kauf durchführen.}
\end{tikzpicture}
\hspace{-0.6cm}
\begin{tikzpicture}
	\card
	\cardstrip
	\cardbanner{banner/white.png}
	\cardicon{icons/coin.png}
	\cardprice{4}
	\cardtitle{Bischof}
	\cardcontent{Du erhältst +\coin[1] und legst einen \victorypointtoken-Marker auf dein Spieler-Tableau. Dann musst du eine Handkarte entsorgen, wenn du mindestens 1 Karte auf der Hand hast. Lege halb so viele \victorypointtoken-Marker auf dein Spieler-Tableau, wie die entsorgte Karte gekostet hat. Ungerade Kosten werden abgerundet. Die \potion-Kosten (z.B. aus \emph{Alchemie}) spielen keine Rolle. So erhältst du je 2 \victorypointtoken-Marker für ein entsorgtes \emph{ARBEITERDORF} (Kosten: \coin[4]), ebenso wie für ein \emph{GESINDEL} (Kosten: \coin[5]) oder einen \emph{GOLEM} (Kosten: \coin[4] und \potion). 
	\
	\medskip

	Jeder Mitspieler darf eine Handkarte entsorgen, erhält dafür aber keine Siegpunktmarker.}
\end{tikzpicture}
\hspace{-0.6cm}
\begin{tikzpicture}
	\card
	\cardstrip
	\cardbanner{banner/white.png}
	\cardicon{icons/coin.png}
	\cardprice{4}
	\cardtitle{Denkmal}
	\cardcontent{Du erhältst +\coin[2] und legst einen \victorypointtoken-Marker aus dem Vorrat auf dein Spieler-Tableau.}
\end{tikzpicture}
\hspace{-0.6cm}
\begin{tikzpicture}
	\card
	\cardstrip
	\cardbanner{banner/gold.png}
	\cardicon{icons/coin.png}
	\cardprice{4}
	\cardtitle{Steinbruch}
	\cardcontent{Diese Karte ist eine Geldkarte mit zusätzlichen Anweisungen. Sie hat den Wert \coin[1].

	\medskip

	Solange diese Karte im Spiel ist, kosten alle Aktionskarten (auch kombinierte) \coin[2] weniger. Dies betrifft alle Aktionskarten, d.h. auch Handkarten, Karten in den Ablage- und Nachziehstapeln etc. Der Effekt ist kumulativ, d.h. mit einem zweiten \emph{STEINBRUCH} oder anderen Aktionskarten, die die Kosten von Karten reduzieren, können die Kosten weiter gesenkt werden.}
\end{tikzpicture}
\hspace{-0.6cm}
\begin{tikzpicture}
	\card
	\cardstrip
	\cardbanner{banner/gold.png}
	\cardicon{icons/coin.png}
	\cardprice{4}
	\cardtitle{Talisman}
	\cardcontent{Diese Karte ist eine Geldkarte mit zusätzlichen Anweisungen. Sie hat den  Wert \coin[1].

	\medskip

	Solange diese Karte im Spiel ist und du eine Nicht-Punktekarte kaufst (nicht wenn du sie auf andere Weise nimmst), die zu diesem Zeitpunkt maximal \coin[4] kostet, nimmst du dir eine weitere gleiche Karte vom Vorrat. Ist keine gleiche Karte im Vorrat, nimmst du dir keine weitere Karte. Kaufst du in einem Zug mehrere Karten, die maximal \coin[4] kosten, wendest du den Effekt des \emph{TALISMANS} auf alle diese Karten an.}
\end{tikzpicture}
\hspace{-0.6cm}
\begin{tikzpicture}
	\card
	\cardstrip
	\cardbanner{banner/gold.png}
	\cardicon{icons/coin.png}
	\cardprice{5}
	\cardtitle{Abenteuer}
	\cardcontent{Diese Karte ist eine Geldkarte mit zusätzlichen Anweisungen. Sie hat den Wert \coin[1]. 

	\medskip

	Sobald du diese Karte ausspielst (normalerweise in der Kaufphase), deckst du solange Karten vom Nachziehstapel auf, bis du die erste Geldkarte (auch kombinierte) aufdeckst. Ist der Nachziehstapel aufgebraucht, ohne eine Geldkarte zu finden, mischst du deinen Ablagestapel. Findest du auch dort keine Geldkarte, legst du alle aufgedeckten Karten ab. Spiele die erste aufgedeckte Geldkarte sofort aus und führe ggf. zusätzliche Anweisungen auf dieser Karte aus. Lege alle anderen aufgedeckten Karten ab.}
\end{tikzpicture}
\hspace{-0.6cm}
\begin{tikzpicture}
	\card
	\cardstrip
	\cardbanner{banner/white.png}
	\cardicon{icons/coin.png}
	\cardprice{5}
	\cardtitle{Gesindel}
	\cardcontent{Ziehe drei Karten nach. Anschließend muss jeder Mitspieler (beginnend bei deinem linken Nachbarn) die obersten drei Karten seines Nachziehstapels aufdecken und alle aufgedeckten Geldkarten sowie Aktionskarten (auch kombinierte), ablegen. Alle anderen aufgedeckten Karten legt er in beliebiger Reihenfolge zurück auf den Nachziehstapel.}
\end{tikzpicture}
\hspace{-0.6cm}
\begin{tikzpicture}
	\card
	\cardstrip
	\cardbanner{banner/white.png}
	\cardicon{icons/coin.png}
	\cardprice{5}
	\cardtitle{Gewölbe}
	\cardcontent{Ziehe 2 Karten nach. Lege anschließend beliebig viele Handkarten (auch 0) ab. Du darfst auch Karten ablegen, die du gerade erst nachgezogen hast. Für jede abgelegte Karte erhältst du +\coin[1].

	\medskip

	Jeder Mitspieler darf 2 Handkarten ablegen und eine Karte nachziehen. Falls ein Mitspieler nur 1 Handkarte hat, darf er diese zwar ablegen, jedoch keine Karte nachziehen.}
\end{tikzpicture}
\hspace{-0.6cm}
\begin{tikzpicture}
	\card
	\cardstrip
	\cardbanner{banner/gold.png}
	\cardicon{icons/coin.png}
	\cardprice{6}
	\cardtitle{Hort}
	\cardcontent{Diese Karte ist eine Geldkarte mit zusätzlichen Anweisungen. Sie hat den Wert \coin[2].

	\medskip

	Solange diese Karte im Spiel ist und du eine Punktekarte (auch kombinierte) kaufst, nimmst du ein \emph{GOLD} vom Vorrat. Wenn kein \emph{GOLD} mehr im Vorrat ist, erhältst du nichts. Nimmst du eine Punktekarte auf andere Weise (d.h. nicht durch einen Kauf), nimmst du dir kein \emph{GOLD}. Hast du zwei \emph{HORTE} im Spiel, nimmst du dir pro gekaufter Punktekarte zwei \emph{GOLD} usw. Kaufst du in einem Spielzug zwei oder mehr Punktekarten, nimmst du dir für jede Punktekarte entsprechend viele \emph{GOLD} vom Vorrat. Du erhältst auch \emph{GOLD}, wenn du die gekaufte Punktekarte im gleichen Spielzug wieder entsorgst.}
\end{tikzpicture}
\hspace{-0.6cm}
\begin{tikzpicture}
	\card
	\cardstrip
	\cardbanner{banner/gold.png}
	\cardicon{icons/coin.png}
	\cardprice{5}
	\cardtitle{\tiny{Königliches Siegel}}
	\cardcontent{Diese Karte ist eine Geldkarte mit zusätzlichen Anweisungen. Sie hat den Wert \coin[2].

	\medskip

	Solange diese Karte im Spiel ist, entscheidest du für jede Karte, die du kaufst oder auf andere Weise nimmst, ob du sie ablegen oder oben auf deinen Nachziehstapel legen möchtest.}
\end{tikzpicture}
\hspace{-0.6cm}
\begin{tikzpicture}
	\card
	\cardstrip
	\cardbanner{banner/white.png}
	\cardicon{icons/coin.png}
	\cardprice{5}
	\cardtitle{Leihhaus}
	\cardcontent{Sieh dir deinen kompletten Ablagestapel an und nimm beliebig viele \emph{KUPFER} auf die Hand. Zeige diese vorher deinen Mitspielern. Die restlichen Karten legst du in beliebiger Reihenfolge wieder auf den Ablagestapel.}
\end{tikzpicture}
\hspace{-0.6cm}
\begin{tikzpicture}
	\card
	\cardstrip
	\cardbanner{banner/white.png}
	\cardicon{icons/coin.png}
	\cardprice{5}
	\cardtitle{Münzer}
	\cardcontent{Du darfst eine Geldkarte aus deiner Hand aufdecken. Wenn du das tust, nimm dir eine Karte mit gleichem Namen vom Vorrat. Ist keine entsprechende Karte im Vorrat vorhanden, erhältst du nichts. Nimm die aufgedeckte Geldkarte zurück auf die Hand.

	\medskip

	Wenn du den \emph{MÜNZER} kaufst, musst du alle Geldkarten, die du im Spiel hast, sofort entsorgen. Sofern du nach dem Kauf des \emph{MÜNZERS} noch \coin und einen weiteren Kauf übrig hast, darfst du die Geldwerte der entsorgten Karten in diesem Zug noch verwenden. Wenn du den \emph{MÜNZER} kaufst und eine Geldkarte im Spiel ist, deren zusätzliche Anweisung in Kraft tritt, sobald du eine Karte kaufst (z.B. \emph{KÖNIGLICHES SIEGEL}), wird diese Geldkarte entsorgt, bevor deren Effekt eintreten kann.

	\medskip

	Beachte, dass du alle Geldkarten, die du in deiner Kaufphase verwenden möchtest, vor deinem ersten Kauf ausspielen musst. Du kannst nicht 1 \emph{GOLD} und 1 \emph{SILBER} ausspielen, den \emph{MÜNZER} kaufen, diese Geldkarten entsorgen und dann weiteres Geld auslegen. Sobald eine Karte gekauft wurde, dürfen keinen weiteren Geldkarten mehr ausgespielt werden.}
\end{tikzpicture}
\hspace{-0.6cm}
\begin{tikzpicture}
	\card
	\cardstrip
	\cardbanner{banner/white.png}
	\cardicon{icons/coin.png}
	\cardprice{5}
	\cardtitle{Quacksalber}
	\cardcontent{Du erhältst +\coin[2].

	\medskip

	Jeder Mitspieler (beginnend bei deinem linken Nachbarn) legt entweder einen \emph{FLUCH} aus seiner Hand ab oder er nimmt einen \emph{FLUCH} und ein \emph{KUPFER} vom Vorrat und legt diese ab. Dies dürfen die Spieler auch wählen, wenn der \emph{KUPFER}- oder \emph{FLUCH}-Stapel leer sind. Nimmt ein Spieler den \emph{FLUCH} und das \emph{KUPFER} und hat einen \emph{WACHTURM} auf der Hand, darf er diesen nach jeder genommenen Karte aufdecken (oder wahlweise nur nach einer) und z.B. den \emph{FLUCH} entsorgen und das \emph{KUPFER} auf den Nachziehstapel legen (oder ebenfalls entsorgen).}
\end{tikzpicture}
\hspace{-0.6cm}
\begin{tikzpicture}
	\card
	\cardstrip
	\cardbanner{banner/gold.png}
	\cardicon{icons/coin.png}
	\cardprice{5}
	\cardtitle{\scriptsize{Schmuggelware}}
	\cardcontent{Diese Karte ist eine Geldkarte mit zusätzlichen Anweisungen. Sie hat den Wert \coin[3]. Du erhältst + 1 Kauf.

	\medskip

	Dein linker Mitspieler nennt den Namen einer beliebigen Karte (z.B. \enquote{Provinz}). Diese muss nicht Teil des Vorrats sein. Du darfst diese Karte in diesem Zug nicht kaufen. Wenn du die Karte auf andere Weise nehmen kannst, darfst du das tun. Spielst du mehrere \emph{SCHMUGGELWAREN} aus, nennt dein linker Mitspieler entsprechend viele Karten.}
\end{tikzpicture}
\hspace{-0.6cm}
\begin{tikzpicture}
	\card
	\cardstrip
	\cardbanner{banner/white.png}
	\cardicon{icons/coin.png}
	\cardprice{5}
	\cardtitle{Stadt}
	\cardcontent{Du erhältst + 1 Karte sowie + 2 Aktionen. Wenn noch kein Vorratsstapel leer ist, passiert nichts weiter. Wenn genau 1 Vorratsstapel leer ist, erhältst du nochmal + 1 Karte. Wenn 2 oder mehr Vorratsstapel leer sind, erhältst du stattdessen + 1 Karte, +\coin[1] und + 1 Kauf. }
\end{tikzpicture}
\hspace{-0.6cm}
\begin{tikzpicture}
	\card
	\cardstrip
	\cardbanner{banner/white.png}
	\cardicon{icons/coin.png}
	\cardprice{6}
	\cardtitle{\scriptsize{Großer Markt}}
	\cardcontent{Du erhältst + 1 Karte, + 1 Aktion, + 1 Kauf sowie +\coin[2].

	\medskip

	Wenn du diese Karte kaufen möchtest, darfst du zu diesem Zeitpunkt kein \emph{KUPFER} im Spiel haben. Wenn du zu einem früheren Zeitpunkt in deinem Zug \emph{KUPFER} im Spiel hattest, dieses aber entsorgt hast, darfst du den \emph{GROSSEN MARKT} kaufen. Kannst du den \emph{GROSSEN MARKT} auf andere Art nehmen, darfst du das jederzeit tun, auch wenn du \emph{KUPFER} im Spiel hast.}
\end{tikzpicture}
\hspace{-0.6cm}
\begin{tikzpicture}
	\card
	\cardstrip
	\cardbanner{banner/white.png}
	\cardicon{icons/coin.png}
	\cardprice{7}
	\cardtitle{Ausbau}
	\cardcontent{Entsorge eine beliebige Handkarte und nimm eine Karte vom Vorrat, die bis zu \coin[3] mehr kostet als die entsorgte Karte. Du darfst den Betrag nicht mit zusätzlichem \coin erhöhen. Hast du keine Karte auf der Hand, die du entsorgen kannst, darfst du dir keine Karte vom Vorrat nehmen. Den ausgespielten \emph{AUSBAU} selbst darfst du nicht entsorgen, da er sich nicht mehr auf deiner Hand befindet.}
\end{tikzpicture}
\hspace{-0.6cm}
\begin{tikzpicture}
	\card
	\cardstrip
	\cardbanner{banner/white.png}
	\cardicon{icons/coin.png}
	\cardprice{6}
	\cardtitle{\scriptsize{Halsabschneider}}
	\cardcontent{Du erhältst + 1 Kauf sowie +\coin[2]. Alle Mitspieler müssen Handkarten ablegen, bis sie nur noch 3 Karten auf der Hand haben. Hat ein Mitspieler bereits 3 oder weniger Karten auf der Hand, muss er keine Karten ablegen.

	\medskip

	Solange diese Karte im Spiel ist, legst du immer, wenn du eine Karte kaufst (nicht wenn du sie auf andere Weise nimmst), einen \victorypointtoken-Marker auf dein Spieler-Tableau. Hast du zwei \emph{HALSABSCHNEIDER} im Spiel, legst du zwei \victorypointtoken-Marker pro gekaufter Karte auf dein Tableau usw.}
\end{tikzpicture}
\hspace{-0.6cm}
\begin{tikzpicture}
	\card
	\cardstrip
	\cardbanner{banner/gold.png}
	\cardicon{icons/coin.png}
	\cardprice{7}
	\cardtitle{Bank}
	\cardcontent{Diese Karte ist eine Geldkarte mit einem variablen Wert: Pro Geldkarte (inklusive dieser / auch kombinierte Geldkarten), die du im Spiel hast, ist sie \coin[1] wert. Spielst du die \emph{BANK} als erste Geldkarte in deinem Zug aus, ist sie genau \coin[1] wert. Spielst du dagegen zuerst ein \emph{GOLD}, ein \emph{SILBER} und zwei \emph{KUPFER} und dann die \emph{BANK}, ist die \emph{BANK} \coin[5] wert. Spielst du im Anschluss noch eine \emph{BANK} aus, bleibt die erste \emph{BANK} \coin[5] wert, die zweite \coin[6].}
\end{tikzpicture}
\hspace{-0.6cm}
\begin{tikzpicture}
	\card
	\cardstrip
	\cardbanner{banner/white.png}
	\cardicon{icons/coin.png}
	\cardprice{7}
	\cardtitle{Königshof}
	\cardcontent{Diese Karte ist ähnlich dem \emph{THRONSAAL} aus dem Basisspiel – mit dem Unterschied, dass du die Aktionskarte, die du aus deiner Hand wählst, dreimal (statt zweimal) ausspielst. Lege die gewählte Aktionskarte aus, führe die Anweisungen darauf komplett aus, nimm sie zurück auf die Hand, spiele sie erneut aus, führe die Anweisungen darauf komplett aus, nimm sie zurück auf die Hand und spiele sie ein drittes Mal aus und führe die Anweisungen darauf komplett aus. Für das dreimalige Ausspielen der Aktionskarte benötigst du keine Aktionen. Du darfst zwischen dem dreimaligen Ausspielen der Aktionskarte keine andere Aktion ausspielen, außer die Aktionskarte selbst gibt dazu die Anweisung.  }
\end{tikzpicture}
\hspace{-0.6cm}
\begin{tikzpicture}
	\card
	\cardstrip
	\cardbanner{banner/white.png}
	\cardicon{icons/coin.png}
	\cardprice{7}
	\cardtitle{\footnotesize{Kunstschmiede}}
	\cardcontent{Egal ob du keine Karte entsorgst (\coin[0] insgesamt) oder z.B. drei Karten, die jeweils \coin[2] kosten (\coin[6] insgesamt) – du \emph{musst} eine Karte vom Vorrat nehmen, die genau so viel kostet wie die entsorgten Karten zusammen gekostet haben, außer es ist keine entsprechende Karte im Vorrat vorhanden. Entsorgst du keine Karten und ist beispielsweise der \emph{KUPFER}-Stapel leer, musst du dir u.U. (wenn keine anderen Karten mit \coin[0]-Kosten vorhanden sind) einen \emph{FLUCH} vom Vorrat nehmen, der ebenfalls \coin[0] kostet. \potion-Kosten für Karten aus Alchemie haben für die \emph{KUNSTSCHMIEDE} keine Auswirkung. Es darf auch keine Karte, die \potion-Kosten enthält, genommen werden.}
\end{tikzpicture}
\hspace{-0.6cm}
\begin{tikzpicture}
	\card
	\cardstrip
	\cardbanner{banner/white.png}
	\cardicon{icons/coin.png}
	\cardprice{8*}
	\cardtitle{Hausierer}
	\cardcontent{Diese Karte ist eine Karte mit variablen Kosten (vgl. NEUE REGELN; S. 6). Du erhältst + 1 Karte, + 1 Aktion sowie +\coin[1]. 

	\medskip

	Wenn du diese Karte in deiner Kaufphase kaufst, kostet sie für jede Aktionskarte, die du im Spiel hast, \coin[2] weniger, niemals allerdings weniger als \coin[0]. Kaufst du eine Karte außerhalb der Kaufphase (z.B. durch den \emph{SCHWARZMARKT}), kostet der HAUSIERER \coin[8], egal ob du weitere Aktionskarten im Spiel hast oder nicht. Aktionskarten, die durch den \emph{THRONSAAL} (aus Basisspiel) oder den \emph{KÖNIGSHOF} mehrfach ausgespielt wurden, sind trotzdem jeweils nur einmal im Spiel und reduzieren die Kosten eines \emph{HAUSIERERS} um \coin[2].}
\end{tikzpicture}
\hspace{-0.6cm}
\begin{tikzpicture}
	\card
	\cardstrip
	\cardbanner{banner/gold.png}
	\cardicon{icons/coin.png}
	\cardprice{9}
	\cardtitle{Platin}
	\cardcontent{Diese Karte ist eine Basiskarte und keine Königreichkarte. Spielt ihr ausschließlich mit Königreichkarten aus Blütezeit, wird diese Karte zusätzlich zu den Basis-Geldkarten \emph{KUPFER}, \emph{SILBER} und \emph{GOLD} in der Spielvorbereitung in den Vorrat gelegt. Bei Spielen mit Königreichkarten aus verschiedenen Editionen oder Erweiterungen entscheidet vor Spielbeginn, ob ihr \emph{PLATIN} in den Vorrat legen wollt oder nicht (vgl. SPIELVORBEREITUNG, S. 4).}
\end{tikzpicture}
\hspace{-0.6cm}
\begin{tikzpicture}
	\card
	\cardstrip
	\cardbanner{banner/green.png}
	\cardicon{icons/coin.png}
	\cardprice{11}
	\cardtitle{Kolonien}
	\cardcontent{Diese Karte ist eine Basiskarte und keine Königreichkarte. Spielt ihr ausschließlich mit Königreichkarten aus Blütezeit, wird diese Karte zusätzlich zu den Basis-Punktekarten \emph{ANWESEN}, \emph{HERZOGTUM} und \emph{PROVINZ} in der Spielvorbereitung in den Vorrat gelegt. Bei Spielen mit Königreichkarten aus verschiedenen Editionen oder Erweiterungen entscheidet vor Spielbeginn, ob ihr die \emph{KOLONIE} in den Vorrat legen wollt oder nicht. Achtet darauf, dass in diesem Fall das Spiel auch endet, wenn der Vorratsstapel \emph{KOLONIE} leer ist (vgl. SPIELVORBEREITUNG, S. 4 sowie ALTERNATIVES SPIELENDE, S. 6).}
\end{tikzpicture}
\hspace{-0.6cm}
\begin{tikzpicture}
	\card
	\cardstrip
	\cardbanner{banner/white.png}
	\cardtitle{\scriptsize{Empfohlene 10er Sätze\qquad}}
	\cardcontent{\emph{Anfänger:}\\
	Abenteuer, Arbeiterdorf, Ausbau, Bank, Denkmal, Gesindel, Halsabschneider, Königliches Siegel, Leihhaus, Wachturm

	\smallskip

	\emph{Freundliche Interaktion:}\\
	Arbeiterdorf, Bischof, Gewölbe, Handelsroute, Hausierer, Hort, Königliches Siegel, Kunstschmiede, Schmuggelware, Stadt 

	\smallskip

	\emph{Große Aktionen:}\\
	Ausbau, Gesindel, Gewölbe, Großer Markt, Königshof, Lohn, Münzer, Stadt, Steinbruch, Talisman

	\smallskip

	\emph{Haufenweise Geld} (Blütezeit + \textit{Basisspiel}):\\
	Abenteuer, Bank, Großer Markt, Königliches Siegel, Münzer, \textit{Abenteurer}, \textit{Geldverleiher}, \textit{Laboratorium}, \textit{Mine}, \textit{Spion}

	\smallskip

	\emph{Die Armee des Königs} (Blütezeit + \textit{Basisspiel}):\\
	Ausbau, Gesindel, Gewölbe, Handlanger, Königshof, \textit{Bürokrat}, \textit{Burggraben}, \textit{Dorf}, \textit{Ratsversammlung}, \textit{Spion}

	\smallskip

	\emph{Ein gutes Leben:} (Blütezeit + \textit{Basisspiel}):\\
	Denkmal, Hort, Leihhaus, Quacksalber, Schmuggelware, \textit{Bürokrat}, \textit{Dorf}, \textit{Gärten}, \textit{Kanzler}, \textit{Keller}}
\end{tikzpicture}
\hspace{-0.6cm}
\begin{tikzpicture}
	\card
	\cardstrip
	\cardbanner{banner/white.png}
	\cardtitle{\scriptsize{Empfohlene 10er Sätze\qquad}}
	\cardcontent{\emph{Pfade zum Sieg} (Blütezeit + \textit{Die Intrige}):\\
	Bischof, Denkmal, Halsabschneider, Hausierer, Leihhaus, \textit{Anbau}, \textit{Armenviertel}, \textit{Baron}, \textit{Handlanger}, \textit{Harem}

	\smallskip

	\emph{All along the watchtower} (Blütezeit + \textit{Die Intrige}):\\
	Gewölbe, Handelsroute, Hort, Talisman, Wachturm, \textit{Bergwerk}, \textit{Brücke}, \textit{Große Halle}, \textit{Handlanger}, \textit{Kerkermeister}

	\smallskip

	\emph{Glücksritter} (Blütezeit + \textit{Die Intrige}):\\
	Ausbau, Bank, Gewölbe, Königshof, Kunstschmiede, \textit{Brücke}, \textit{Kupferschmied}, \textit{Tribut}, \textit{Trickser}, \textit{Wunschbrunnen}}
\end{tikzpicture}
\hspace{-0.6cm}
\begin{tikzpicture}
	\card
	\cardstrip
	\cardbanner{banner/white.png}
	\cardtitle{Platzhalter\quad}
\end{tikzpicture}
\hspace{0.6cm}
