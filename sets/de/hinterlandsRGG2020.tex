% Basic settings for this card set
\renewcommand{\cardcolor}{hinterlands}
\renewcommand{\cardextension}{Erweiterung V}
\renewcommand{\cardextensiontitle}{Hinterland}
\renewcommand{\seticon}{hinterlands.png}

\clearpage
\newpage
\section{\cardextension \ - \cardextensiontitle \ (Rio Grande Games 2020)}

\begin{tikzpicture}
	\card
	\cardstrip
	\cardbanner{banner/white.png}
	\cardicon{icons/coin.png}
	\cardprice{2}
	\cardtitle{Herzogin}
	\cardcontent{Wenn du diese Karte spielst, erhältst du +\coin[2]. Dann schaut jeder Spieler, beginnend mit dir, die oberste Karte seines Nachziehstapels an und entscheidet für sich selbst, ob er diese ablegt oder zurück auf den Nachziehstapel legt.

	\smallskip

	Ist der \emph{HERZOGINEN}-Stapel Teil des Königreichs, darf jeder Spieler, der ein \emph{HERZOGTUM} nimmt, zusätzlich eine \emph{HERZOGIN} nehmen -- solange ihr Stapel nicht leer ist.}
\end{tikzpicture}
\hspace{-0.6cm}
\begin{tikzpicture}
	\card
	\cardstrip
	\cardbanner{banner/goldblue.png}
	\cardicon{icons/coin.png}
	\cardprice{2}
	\cardtitle{Katzengold}
	\cardcontent{Diese Karte ist sowohl eine Geldkarte als auch eine Reaktionskarte. Wenn du sie als Geldkarte spielst, ist \emph{KATZENGOLD} beim ersten Mal in einem Zug \coin[1] wert, bei jedem weiteren Mal in diesem Zug \coin[4]. Spielt man zum Beispiel mit einem \emph{FALSCHGELD} ein \emph{KATZENGOLD} zweimal, bekommt man \coin[1] und \coin[4], also insgesamt \coin[5], obwohl es beide Male das erste ausgespielte Katzengold ist.
	
	\smallskip
	
	Wenn du ein \emph{KATZENGOLD} auf deiner Hand hast und ein Mitspieler eine \emph{PROVINZ} nimmt, darfst du mit dem \emph{KATZENGOLD} reagieren: Du entsorgst das \emph{KATZENGOLD} und nimmst aus dem Vorrat ein \emph{GOLD}, das du verdeckt auf deinen Nachziehstapel legst.}
\end{tikzpicture}
\hspace{-0.6cm}
\begin{tikzpicture}
	\card
	\cardstrip
	\cardbanner{banner/white.png}
	\cardicon{icons/coin.png}
	\cardprice{2}
	\cardtitle{\footnotesize{Wegkreuzung}}
	\cardcontent{Decke deine Handkarten auf. Für jede Punktekarte (auch ggf. kombinierte) erhältst du +1 Karte. Wenn du die \emph{WEGKREUZUNG} zum ersten Mal in diesem Zug ausspielst, erhältst du +3 Aktionen. Für jede weitere ausgespielte \emph{WEGKREUZUNG} in diesem Zug erhältst du keine weiteren Aktionen.}
\end{tikzpicture}
\hspace{-0.6cm}
\begin{tikzpicture}
	\card
	\cardstrip
	\cardbanner{banner/white.png}
	\cardicon{icons/coin.png}
	\cardprice{3}
	\cardtitle{Aufbau}
	\cardcontent{Entsorge eine beliebige deiner Handkarten. Wenn du eine Karte entsorgt hast, nimm dir zwei Karten vom Vorrat: 1 Karte, die genau \coin[1] mehr kostet als die entsorgte Karte, und 1 Karte, die genau \coin[1] weniger kostet als die entsorgte Karte. Kannst du dir eine bzw. beide Karten nicht nehmen, weil keine entsprechende Karte im Vorrat ist, erhältst du nur die verfügbare Karte bzw. nichts. Lege die genommene Karte in beliebiger Reihenfolge auf deinen Nachziehstapel.}
\end{tikzpicture}
\hspace{-0.6cm}
\begin{tikzpicture}
	\card
	\cardstrip
	\cardbanner{banner/white.png}
	\cardicon{icons/coin.png}
	\cardprice{3}
	\cardtitle{Komplott}
	\cardcontent{In deiner Aufräumphase darfst du für jedes Mal, dass du in diesem Zug ein \emph{KOMPLOTT} gespielt hast, eine im Spiel befindliche Aktionskarte, die du normalerweise abgelegen müsstest, stattdessen verdeckt auf deinen Nachziehstapel legen. Das darf auch das \emph{KOMPLOTT} selbst sein.}
\end{tikzpicture}
\hspace{-0.6cm}
\begin{tikzpicture}
	\card
	\cardstrip
	\cardbanner{banner/white.png}
	\cardicon{icons/coin.png}
	\cardprice{3}
	\cardtitle{Oase}
	\cardcontent{Du legst eine Handkarte ab -- das darf auch due gerade gezogene Karte sein. Auch wenn du keine Karte nachziehen kannst, legst du trotzdem eine Handkarte ab.}
\end{tikzpicture}
\hspace{-0.6cm}
\begin{tikzpicture}
	\card
	\cardstrip
	\cardbanner{banner/white.png}
	\cardicon{icons/coin.png}
	\cardprice{3}
	\cardtitle{Orakel}
	\cardcontent{Jeder Spieler, beginnend bei dir, deckt die beiden obersten Karten seines Nachziehstapels auf und du entscheidest für jeden, ob er sie ablegt oder auf seinen Nachziehstapel zurücklegt. Die Reihenfolge, in der ein Spieler die Karten auf seinen Nachziehstapel legt, bestimmt er selbst. Danach ziehst du 2 Karten.}
\end{tikzpicture}
\hspace{-0.6cm}
\begin{tikzpicture}
	\card
	\cardstrip
	\cardbanner{banner/greenblue.png}
	\cardicon{icons/coin.png}
	\cardprice{3}
	\cardtitle{Tunnel}
	\cardcontent{Diese Karte ist sowohl eine Punktekarte als auch eine Reaktionskarte. Wenn du durch eine Anweisung auf einer Karte (egal ob in deinem Zug oder dem Zug eines Mitspielers) gezwungen wirst, eine Karte außerhalb der normalen Aufräumphase abzulegen und du dich für einen \emph{TUNNEL} entscheidest, darfst du diesen aufdecken, bevor du ihn ablegst. Wenn du das tust, nimm ein \emph{GOLD} vom Vorrat und lege dieses auf deinen Ablagestapel.}
\end{tikzpicture}
\hspace{-0.6cm}
\begin{tikzpicture}
	\card
	\cardstrip
	\cardbanner{banner/white.png}
	\cardicon{icons/coin.png}
	\cardprice{4}
	\cardtitle{Edler Räuber}
	\cardcontent{Sobald du diese Karte spielst \emph{oder} sie kaufst, müssen, beginnend bei deinem linken Nachbarn, alle Mitspieler die obersten zwei Karten ihres Nachziehstapels aufdecken. Wer 1 \emph{SILBER} oder 1 \emph{GOLD} aufdeckt, muss dieses entsorgen. Wer 1 \emph{SILBER} und 1 \emph{GOLD} oder 2 \emph{SILBER} oder 2 \emph{GOLD} aufdeckt, muss eine Karte (nach deiner Wahl) entsorgen. Die anderen Karten legt er ab. Wer kein \emph{SILBER} oder \emph{GOLD} (aber mindestens 1 andere Geldkarte, auch ggf. kombinierte) aufdeckt, legt beide Karten ab. Wer gar keine Geldkarte aufdeckt, legt beide Karten ab und nimmt sich ein \emph{KUPFER} vom Vorrat. Nimm alle entsorgten \emph{SILBER} und \emph{GOLD} nd lege sie auf deinen Ablagestapel.
	\linebreak
	Wenn du den \emph{EDLEN RÄUBER} kaufst, dürfen deine Mitspieler \emph{keine} Reaktionskarten, wie z.B. den Burggraben (aus dem \emph{Basisspiel}), die auf das Ausspielen einer Angriffskarte reagieren, aufdecken.}
\end{tikzpicture}
\hspace{-0.6cm}
\begin{tikzpicture}
	\card
	\cardstrip
	\cardbanner{banner/blue.png}
	\cardicon{icons/coin.png}
	\cardprice{4}
	\cardtitle{\tiny{Fahrender Händler}}
	\cardcontent{Wenn du diese Karte spielst, musst du eine deiner Handkarten entsorgen. Pro \coin[1], das die entsorgte Karte kostet, nimmst du dir ein \emph{SILBER} vom Vorrat. So nimmst du dir für eine Karte, die (in diesem Moment) \coin[3] kostet, drei \emph{SILBER} vom Vorrat. Kannst du keine Karte entsorgen, erhältst du auch kein \emph{SILBER}. Kostet eine Karte z.B. \coin[2] und \potion (aus \emph{Alchemisten}), nimmst du dir 2 \emph{SILBER}. \potion-Kosten haben keinen Einfluss.
	
	\smallskip
	
	Wenn du den \emph{FAHRENDEN HÄNDLER} als Reaktion einsetzt, weil du (egal ob in deinem Zug oder dem Zug eines Mitspielers) eine Karte nehmen müsstest, deckst du diese Karte aus deiner Hand auf. Wenn du das tust, nimm dir statt der Karte, die du eigentlich nehmen müsstest, ein \emph{SILBER}. Dieses legst du -- außer eine andere Anweisung (z.B. auf einem im Spiel befindlichen \emph{KÖNIGLICHEN SIEGEL} (aus \emph{Blütezeit})) weist an, dass genommene Karten anderswo hingelegt werden müssen -- auf deinen Ablagestapel. Wenn du eine Karte kaufst, du dir stattdessen aber ein \emph{SILBER} nimmst, musst du die Kosten der eigentlich gekauften Karte bezahlen. Döst die eigentlich gekaufte Karte eine Anweisung beim \emph{Kaufen} aus (\enquote{Wenn du diese Karte kaufst...}), musst du diese Anweisung ausführen, bevor du stattdessen das \emph{SILBER} nimmst. Löst die eigentlich gekaufte Karte eine Anweisung beim \emph{Nehmen} aus (\enquote{Wenn du diese Karte nimmst...}), führst du diese Anweisung nicht aus, da du die Karte zwar gekauft, nicht aber genommen hast.}
\end{tikzpicture}
\hspace{-0.6cm}
\begin{tikzpicture}
	\card
	\cardstrip
	\cardbanner{banner/white.png}
	\cardicon{icons/coin.png}
	\cardprice{4}
	\cardtitle{\scriptsize{Gewürzhändler}}
	\cardcontent{Wenn du diese Karte spielst, darfst du eine beliebige \emph{Geldkarte} aus deiner Hand entsorgen. Wenn du eine Karte entsorgst, darfst du entscheiden zwischen: +2 Karten und +1 Aktion \emph{oder} +\coin[2] und +1 Kauf. Wenn du keine Karte entsorgen kannst oder willst, führst du keine der angegebenen Anweisungen aus.}
\end{tikzpicture}
\hspace{-0.6cm}
\begin{tikzpicture}
	\card
	\cardstrip
	\cardbanner{banner/white.png}
	\cardicon{icons/coin.png}
	\cardprice{4}
	\cardtitle{\scriptsize{Lebenskünstler}}
	\cardcontent{Wenn du diese Karte spielst, führst du die 4 Anweisungen der Reihe nach aus:
	\begin{enumerate}
		\item Nimm ein \emph{SILBER} vom Vorrat und lege es auf deinen Ablagestapel.
		\item Sieh dir die oberste Karte deines Nachziehstapels an. Du entscheidest, ob du diese Karte ablegst oder auf den Nachziehstapel zurücklegst.
		\item Ziehe solange Karten nach, bis du 5 Karten auf der Hand hast. Hast du bereits 5 oder mehr Karten auf der Hand, ziehst du keine Karten nach.
		\item Du \emph{darfst} eine Karte, die kiene Geldkarte (auch keine kombinierte) ist, aus deiner Hand entsorgen.
	\end{enumerate}}
\end{tikzpicture}
\hspace{-0.6cm}
\begin{tikzpicture}
	\card
	\cardstrip
	\cardbanner{banner/white.png}
	\cardicon{icons/coin.png}
	\cardprice{4}
	\cardtitle{\footnotesize{Nomadencamp}}
	\cardcontent{Wenn du das \emph{NOMADENCAMP} kaufst oder anderweitig nimmst, legst du es auf deinen Nachziehstapel, statt es wie üblich auf deinen Ablagestapel zu legen.}
\end{tikzpicture}
\hspace{-0.6cm}
\begin{tikzpicture}
	\card
	\cardstrip
	\cardbanner{banner/green.png}
	\cardicon{icons/coin.png}
	\cardprice{4}
	\cardtitle{\footnotesize{Seidenstrasse}}
	\cardcontent{Diese Karte ist eine Punktekarte und hat bis zum Ende des Spiels keine Funktion. Bei Spielende zählt jeder Spieler alle Punktekarten (inkl. dieser \emph{SEIDENSTRASSE}) in seinem Kartensatz (Nachziehstapel, Handkarten und Ablagestapel sowie evtl. zur Seite gelegte Karten). Diese Anzahl durch 4 dividiert (eventueller Rest bleibt unberücksichtigt) ist der Wert jeder \emph{SEIDENSTRASSE} des Spielers. So erhält ein Spieler mit 4 Punktekarten genauso 1 \victorypoint wie ein Spieler mit 7 Punktekarten, Spieler mit 8-11 Punktekarten erhalten 2 \victorypoint usw.}
\end{tikzpicture}
\hspace{-0.6cm}
\begin{tikzpicture}
	\card
	\cardstrip
	\cardbanner{banner/gold.png}
	\cardicon{icons/coin.png}
	\cardprice{5}
	\cardtitle{Blutzoll}
	\cardcontent{Diese Karte ist eine Geldkarte mit dem Wert \coin[1]. Wenn du den \emph{BLUTZOLL} spielst, darfst du ein \emph{KUPFER} vom Vorrat direkt auf deine Hand nehmen. Du darfst diese \emph{KUPFER} noch im gleichen Zug spielen. Wenn du den \emph{BLUTZOLL} kaufst oder anderweitig nimmst, muss sich, beginnend bei deinem linken Nachbarn, jeder Mitspieler einen \emph{FLUCH} nehmen. Sind nicht genug Fluchkarten im Vorrat, werden nur die vorhandenen in Spielreihenfolge verteilt.}
\end{tikzpicture}
\hspace{-0.6cm}
\begin{tikzpicture}
	\card
	\cardstrip
	\cardbanner{banner/white.png}
	\cardicon{icons/coin.png}
	\cardprice{5}
	\cardtitle{Botschaft}
	\cardcontent{Wenn du diese Karte spielst, ziehst du zuerst 5 Karten. Lege danach 3 Handkarten ab. Du kannst auch Karten ablegen, die du gerade gezogen hast.
	
	\medskip
	
	Wenn du diese Karte nimmst, nimmt sich jeder Mitspieler, beginnend bei deinem linken Nachbarn, ein \emph{SILBER} vom Vorrat. Ist nicht genug \emph{SILBER} im Vorrat, wird nur das vorhandene in Spielreihenfolge verteilt.}
\end{tikzpicture}
\hspace{-0.6cm}
\begin{tikzpicture}
	\card
	\cardstrip
	\cardbanner{banner/white.png}
	\cardicon{icons/coin.png}
	\cardprice{5}
	\cardtitle{Feilscher}
	\cardcontent{Wenn sich diese Karte im Spiel befindet und du eine beliebige Karte \emph{kaufst}, musst du dir zusätzlich zu der gekauften Karte eine Karte vom Vorrat nehmen, die weniger kostet als die gekaufte Karte und die keine Punktekarte ist. Lege die zusätzlich genommene Karte auf deinen Ablagestapel. Gibt es keine Karte im Vorrat, die weniger kostet als die gekaufte Karte (z.B. weil die gekaufte Karte \coin[0] kostet), dann nimmst du keine zusätzliche Karte. Nimmst du eine Karte auf andere Weise (d.h. wenn du sie nicht kaufst), nimmst du dir keine zusätzliche Karte. Der \emph{FEILSCHER} muss im Spiel sein, damit du dir eine zusätzliche Karte nehmen kannst. Sind mehrere \emph{FEILSCHER} im Spiel, musst du entsprechend mehr Karten vom Vorrat nehmen; zum Beispiel bei 3 \emph{FEILSCHERN} im Spiel musst du dir 3 Karten vom Vorrat nehmen.}
\end{tikzpicture}
\hspace{-0.6cm}
\begin{tikzpicture}
	\card
	\cardstrip
	\cardbanner{banner/white.png}
	\cardicon{icons/coin.png}
	\cardprice{5}
	\cardtitle{Fernstrasse}
	\cardcontent{Solange diese Karte im Spiel ist, werden die Kosten aller Karten (d.h. Handkarten, Karten aus dem Nachzieh- und Ablagestapeln aller Spieler, Karten im Vorrat, evtl. beiseite gelegte Karten etc.) im \coin[1] reduziert, niemals jedoch auf weniger als \coin[0]. Der Effekt ist kumulativ, d.h. sind z.B. 2 \emph{FERNSTRASSEN} im Spiel, reduzieren sich die Kosten der Karten um \coin[2]. Wenn du mit einem \emph{THRONSAAL} (aus dem \emph{Basisspiel}) die gleiche \emph{FERNSTRASSE} zweimal ausspielst, ist sie trotzdem nur einmal im Spiel und reduziert die Kosten aller Karten nur um \coin[1].}
\end{tikzpicture}
\hspace{-0.6cm}
\begin{tikzpicture}
	\card
	\cardstrip
	\cardbanner{banner/white.png}
	\cardicon{icons/coin.png}
	\cardprice{5}
	\cardtitle{Gasthaus}
	\cardcontent{Wenn du diese Karte kaufst oder auf andere Art und Weise nimmst, darfst du deinen Ablagestapel anschauen. Decke dann beliebig viele Aktionskarten aus deinem Ablagestapel auf (auch diese) und mische alle aufgedeckten Karten in deinen Nachziehstapel. Sollte dein Nachziehstapel gerade leer sein, wenn du den Ablagestapeldurchsiehst, werden die ausgewählten Aktionskartenzu den einzigen Karten deines Nachziehstapels.}
\end{tikzpicture}
\hspace{-0.6cm}
\begin{tikzpicture}
	\card
	\cardstrip
	\cardbanner{banner/white.png}
	\cardicon{icons/coin.png}
	\cardprice{5}
	\cardtitle{Kartograph}
	\cardcontent{Schau dir die obersten 4 Karten deines Nachziehstapels an. Lege beliebig viele davon ab. Die restlichen legst du in beliebiger Reihenfolge zurück auf deinen Nachziehstapel.}
\end{tikzpicture}
\hspace{-0.6cm}
\begin{tikzpicture}
	\card
	\cardstrip
	\cardbanner{banner/white.png}
	\cardicon{icons/coin.png}
	\cardprice{5}
	\cardtitle{Mandarin}
	\cardcontent{Wenn du diese Karte spielst, erhältst du +\coin[3] und legst eine beliebige Handkarte verdeckt auf den Nachziehstapel. Hast du keine Karte auf der Hand, legst du keine Karte auf deinen Nachziehstapel.
	
	\medskip
	
	Wenn du diese Karte kaufst oder auf andere Art und Weise nimmst, legst du alle Geldkarten, die du zu diesem Zeitpunkt im Spiel hast, in beliebiger Reihenfolge verdeckt auf deinen Nachziehstapel. Solltest du den Wert der zurückgelegten Geldkarten in dem Moment, wenn du sieh zurücklegst, noch nicht (oder noch nicht vollständig) genutzt haben, steht dir der entsprechende Geldwert weiterhin (z.B. für deine Kaufphase) zur Verfügung, obwohl du die Karte bereits zurückgelegt hast.}
\end{tikzpicture}
\hspace{-0.6cm}
\begin{tikzpicture}
	\card
	\cardstrip
	\cardbanner{banner/white.png}
	\cardicon{icons/coin.png}
	\cardprice{5}
	\cardtitle{Markgraf}
	\cardcontent{Jeder Mitspieler, beginnend mit deinem linken Nachbarn, zieht eine Karte von seinem Nachziehstapel und legt dann solange Karten ab, bis er maximal 3 Karten auf seiner Hand hat. Hat ein Spieler, auch nachdem er 1 Karte gezogen hat, 3 Karten oder weniger auf seiner Hand, muss er keine Karte ablegen.}
\end{tikzpicture}
\hspace{-0.6cm}
\begin{tikzpicture}
	\card
	\cardstrip
	\cardbanner{banner/gold.png}
	\cardicon{icons/coin.png}
	\cardprice{5}
	\cardtitle{Schatztruhe}
	\cardcontent{Diese Karte ist eine Geldkarte mit dem Wert \coin[3]. Wenn du die \emph{SCHATZTRUHE} kaufst oder auf eine andere Art und Weise nimmst, \emph{musst} du zusätzlich zwei \emph{KUPFER} vom Vorrat nehmen.}
\end{tikzpicture}
\hspace{-0.6cm}
\begin{tikzpicture}
	\card
	\cardstrip
	\cardbanner{banner/white.png}
	\cardicon{icons/coin.png}
	\cardprice{5}
	\cardtitle{Stallungen}
	\cardcontent{Du darfst eine beliebige Geldkarte aus deiner Hand ablegen. Wenn du das tust, ziehst du 3 Karten und erhältst +1 Aktion. Wenn du das nicht tust, darfst du keine Karten ziehen und erhältst keine zusätzliche Aktion.}
\end{tikzpicture}
\hspace{-0.6cm}
\begin{tikzpicture}
	\card
	\cardstrip
	\cardbanner{banner/green.png}
	\cardicon{icons/coin.png}
	\cardprice{6}
	\cardtitle{\tiny{Fruchtbares Land}}
	\cardcontent{Diese Karte ist eine Punktekarte. Bei Spielendeerhält der Spieler, der die Karte in seinem Kartensatz hat, 2 \victorypoint.
	
	\medskip
	
	Wenn du diese Karte kaufst (nicht, wenn du sie auf andere Art und Weise erhältst oder nimmst), \emph{musst} du eine Karte aus deiner Hand entsorgen. Nimm dir eine Karte vom Vorrat, die genau \coin[2] mehr kostet als die entsorge Karte. Kannst du keine Karte entsorgen oder befindet sich keine Karte im Vorrat, die genau \coin[2] mehr kostet, nimmst du dir keine zusätzliche Karte.}
\end{tikzpicture}
\hspace{-0.6cm}
\begin{tikzpicture}
	\card
	\cardstrip
	\cardbanner{banner/white.png}
	\cardicon{icons/coin.png}
	\cardprice{6}
	\cardtitle{Grenzdorf}
	\cardcontent{Wenn du diese Karte spielst, ziest du 1 Karte und erhältst +2 Aktionen. Wenn du diese Karte kaufst oder auf andere Art und Weise nimmst, nimmst du dir eine weitere Karte vom Vorrat, die weniger kostet als das \emph{GRENZDORF}.}
\end{tikzpicture}
\hspace{-0.6cm}
\begin{tikzpicture}
	\card
	\cardstrip
	\cardbanner{banner/white.png}
	\cardtitle{\scriptsize{Spielvorbereitung}\qquad}
	\cardcontent{Zum Spielen benötigt Ihr ein \emph{DOMINION}-Basisspiel oder das Basiskarten-Set. Legt alle Basiskarten wie gewohnt als Vorrats in die Tischmitte.
	\linebreak
	Nutzt ihr die Punktekarten \emph{SEIDENSTRASSE}, \emph{FRUCHTBARES LAND} oder den \emph{TUNNEL}, beachtet bitte die -- von der Spieleranzahl abhängige -- Kartenanzahl.
	
	\medskip

	\begin{tabbing}
		\emph{Bei 3 oder 4 Spielern:}xxx \= jeweils 12 Karten \kill
		\emph{Bei 3 oder 4 Spielern:} \> jeweils 12 Karten \\
		\emph{Bei 2 Spielern:} \> jeweils 8 Karten \\
	\end{tabbing}}
\end{tikzpicture}
\hspace{-0.6cm}
\begin{tikzpicture}
	\card
	\cardstrip
	\cardbanner{banner/white.png}
	\cardtitle{\scriptsize{Neue Regeln (1/2)}\qquad}
	\cardcontent{
		\emph{Es gelten die Basisspielregeln mit folgenden Ergänzungen:}
		
		\medskip
		
		\emph{Geldkarten mit zusätzlichen Anweisungen:} Die Geldkarten \emph{KATZENGOLD}, \emph{BLUTZOLL} und \emph{SCHATZTRUHE} werden, wie alle Geldkarten, in der entsprechenden Spielphase eingesetzt. Darüber hinaus beinhalten diese Karten zusätzliche Anweisungen, die beim Ausspielen beachtet werden müssen (siehe hierzu die entsprechenden Kartenbeschreibungen).
		
		\medskip

		\emph{Kombinierte Königreichkarten:} Die Karten \emph{KATZENGOLD}, \emph{FAHRENDER HÄNDLER} und \emph{TUNNEL} sind kombinierte Karten und haben die Funktion beider angegebener Kartentypen, d.h. sie sind jeweils sowohl eine Reaktionskarte, die während des Spiels als Reaktion genutzt werden kann, als auch eine klassische Punkte- bzw. Geldkarte. Wird die Karte als Reaktionskarte eingesetzt, gilt die Anweisung unterhalb der Trennlinie. Anweisungen auf anderen Aktionskarten, die sich auf Reaktions-, Geld- der Punktekarte beziehen, betreffen auch die jeweiligen kombinierten Karten.
	}
\end{tikzpicture}
\hspace{-0.6cm}
\begin{tikzpicture}
	\card
	\cardstrip
	\cardbanner{banner/white.png}
	\cardtitle{\scriptsize{Neue Regeln (2/2)}\qquad}
	\cardcontent{
		\emph{In der Kaufphase:} Alle Geldkarten, die in einem Zug eingesetzt werden sollen, müssen einzeln nacheinander und \emph{vor dem ersten} Kauf gespielt werden. Der Spieler darf in der Regel \emph{keine weiteren} Geldkarten spielen, nachdem ein Kauf getätigt wurde. Wenn er eine Geldkarte mit zusätzlichen Anweisungen spielt, muss er die Anweisungen zunächst (soweit möglich) ausführen, bevor er eine weiter Geldkarte spielt.
		
		\medskip
		
		\emph{Mehrere Dinge passieren gleichzeitig:} Sollten mehrere Anweisungen gleichzeitig ausgeführt werden können, entscheidet der Spieler selbst, in welcher Reihenfolge er sie ausführt. Sind mehrere Spieler von einer Anweisung betroffen, wird diese in Spielreihenfolge, beginnend bei dem Spieler, der gerade am Zug ist, ausgeführt.
	}
\end{tikzpicture}
\hspace{-0.6cm}
\begin{tikzpicture}
	\card
	\cardstrip
	\cardbanner{banner/white.png}
	\cardtitle{Anweisungen\qquad}
	\cardcontent{
		\emph{Im Spiel:} Geld- und Aktionskarten, die ein Spieler offen in seinem Spielbereich vor sich liegen hat, befinden sich \enquote{im Spiel}, bis sie woanders hingelegt (i.d.R. in der Aufräumphase abgelegt) werden. Nicht \enquote{im Spiel} befinden sich beiseitegelegte und entsorgte Karten, sowie alle Handkarten, Karten im Vorrat und in den Nachzieh- und Ablagestapeln. Auch als Reaktion aufgedeckte Reaktionskarten befinden sich nicht \enquote{im Spiel}.

		\smallskip

		\emph{Wenn du diese/eine Karte nimmst ...:} Sobald ein Spieler eine Karte genommen hat (egal ob in der Aktionsphase, nach einem Kauf oder im Zug eines Mitspielers), führt der Spieler die angegebene Anweisung aus.

		\smallskip

		\emph{Wenn du diese/eine Karte kaufst ...:} Sobald ein Spieler eine Karte gekauft hat (also noch bevor er die gerade gekaufte Karte nimmt), führt der Spieler die angegebene Anweisung aus.

		\smallskip

		\emph{Diese Karte:} Enthält eine Karte eine Anweisung, die sich auf \enquote{diese Karte} bezieht, ist immer die Karte gemeint, auf der die Anweisung steht, niemals eine andere Karte, auf die innerhalb der Anweisung Bezug genommen wird.

		\smallskip

		\emph{Jene Karte:} Enthält eine Karte eine Anweisung, die sich auf \enquote{jene Karte} bezieht, ist immer die Karte gemeint, auf die auf einer Karte (\enquote{dieser Karte}) Bezug genommen wird -- es ist niemals die gerade genutzte Karte gemeint.
	}
\end{tikzpicture}
\hspace{-0.6cm}
\begin{tikzpicture}
	\card
	\cardstrip
	\cardbanner{banner/white.png}
	\cardtitle{\scriptsize{Empfohlene 10er Sätze\qquad}}
	\cardcontent{\emph{Einführung:} \\ 
	Aufbau, Feilscher, Gewürzhändler, Lebenskünstler, Markgraf, Nomadencamp, Oase, Schatztruhe, Stallungen, Wegkreuzung 

	\smallskip 
	
	\emph{Lauterer Wettbewerb:} \\ 
	Aufbau, Blutzoll, Edler Räuber, Fahrender Händler, Fruchtbares Land, Grenzdorf, Herzogin, Kartograph, Seidenstraße, Stallungen 

	\smallskip 
	
	\emph{Gelegenheiten:} \\ 
	Fahrender Händler, Feilscher, Fernstraße, Gewürzhändler, Grenzdorf, Herzogin, Katzengold, Komplott, Nomadencamp, Schatztruhe
	
	\bigskip

	\emph{Straßenräuber} (+ \textit{Basisspiel}):\\ 
	Edler Räuber, Fernstraße, Gasthaus, Markgraf, Oase, \textit{Bibliothek}, \textit{Geldverleiher}, \textit{Keller}, \textit{Thronsaal}, \textit{Werkstatt}

	\smallskip 
	
	\emph{Abenteuerfahrt}  (+ \textit{Basisspiel}):\\ 
	Fruchtbares Land, Gewürzhändler, Katzengold, Orakel, Wegkreuzung, \textit{Abenteurer (bzw. Torwächterin)}, \textit{Jahrmarkt}, \textit{Kanzler (bzw. Vasall)}, \textit{Laboratorium}, \textit{Umbau}}
\end{tikzpicture}
\hspace{-0.6cm}
\begin{tikzpicture}
	\card
	\cardstrip
	\cardbanner{banner/white.png}
	\cardtitle{\scriptsize{Empfohlene 10er Sätze\qquad}}
	\cardcontent{\emph{In der Ferne}  (+ \textit{Dark Ages}):\\ 
	Aufbau, Botschaft, Feilscher, Kartograph, Katzengold, \textit{Barde}, \textit{Bettler}, \textit{Graf}, \textit{Lehen}, \textit{Marodeur}

	\smallskip 
	
	\emph{Expedition}  (+ \textit{Dark Ages}):\\ 
	Fernstraße, Fruchtbares Land, Gewürzhändler, Tunnel, Wegkreuzung, \textit{Altar}, \textit{Armenhaus}, \textit{Eisenhändler}, \textit{Katakomben}, \textit{Lagerraum}
	
	\smallskip
	
	\emph{Einfache Pläne} (+ \textit{Empires}):\\ 
	Aufbau, Blutzoll, Feilscher, Grenzdorf, Stallung, \textit{Labyrinth}, \textit{Spende -- Forum}, \textit{Katapult / Felsen}, \textit{Patrizier / Handelsplatz}, \textit{Tempel}, \textit{Villa}

	\smallskip 
	
	\emph{Expansion}  (+ \textit{Empires}):\\ 
	Fernstraße, Fruchtbares Land, Gewürzhändler, Schatztruhe, Tunnel, \textit{Brunnen}, \textit{Schlachtfeld - Feldlager / Diebesgut}, \textit{Ingenieur}, \textit{Legionär}, \textit{Schlösser}, \textit{Zauber}
	
	\smallskip
	
	\emph{Auf zur Party} (+ \textit{Nocturne}):\\ 
	Fernstraße, Gasthaus, Kartograph, Komplott, Oase, \textit{Druidin (Geschenk des Berges, Gesch. d. Himmels, Gesch. d. Sonne)}, \textit{Getreuer Hund}, \textit{Konklave}, \textit{Schuster}, \textit{Werwolf}

	\smallskip 
	
	\emph{Schafe zählen}  (+ \textit{Nocturne}):\\ 
	Edler Räuber, Feilscher, Fruchtbares Land, Tunnel, Wegkreuzung, \textit{Kobold}, \textit{Krypta}, \textit{Minnesängerin}, \textit{Puka}, \textit{Schäferin}}
\end{tikzpicture}
\hspace{0.6cm}
