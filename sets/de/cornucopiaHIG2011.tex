% Basic settings for this card set
\renewcommand{\cardcolor}{cornucopia}
\renewcommand{\cardextension}{Erweiterung IV}
\renewcommand{\cardextensiontitle}{Reiche Ernte}
\renewcommand{\seticon}{cornucopia.png}

\clearpage
\newpage
\section{\cardextension \ - \cardextensiontitle \ (Hans Im Glück 2011)}

\begin{tikzpicture}
	\card
	\cardstrip
	\cardbanner{banner/white.png}
	\cardicon{icons/coin.png}
	\cardprice{4}
	\cardtitle{Bauerndorf}
	\cardcontent{Du erhältst zunächst +2 Aktionen. Dann deckst du solange Karten von deinem Nachziehstapel auf, bis entweder eine Geldkarte oder eine Aktionskarte offen liegt. Nimm diese Geld- oder Aktionskarte auf die Hand. Lege die übrigen aufgedeckten Karten auf deinen Ablagestapel. Du darfst nicht wählen, ob du eine Geldkarte oder eine Aktionskarte auf die Hand nehmen möchtest. Du musst die erste aufgedeckte Geld- oder Aktionskarte auf die Hand nehmen. Kombinierte Kartentypen sind gleichzeitig alle angegebenen Kartentypen. Du darfst die aufgenommene Karte auch in diesem Zug (nach den üblichen Regeln) ausspielen. Wenn du auch nach dem Mischen deines Ablagestapels keine Geld- oder Aktionskarte aufdecken kannst, nimmst du keine Karte auf die Hand.}
\end{tikzpicture}
\hspace{-0.6cm}
\begin{tikzpicture}
	\card
	\cardstrip
	\cardbanner{banner/white.png}
	\cardicon{icons/coin.png}
	\cardprice{5}
	\cardtitle{Ernte}
	\cardcontent{Decke die obersten 4 Karten von deinem Nachziehstapel auf. Du erhältst +\coin[1] für jede aufgedeckte Karte mit unterschiedlichem Namen. Deckst du z. B. 2 Kupfer, 1 Silber und 1 Anwesen auf, erhältst du +\coin[3]. Kannst du (auch nach dem Mischen des Ablagestapels) nur weniger als 4 Karten aufdecken, deckst du nur so viele auf, wie möglich.}
\end{tikzpicture}
\hspace{-0.6cm}
\begin{tikzpicture}
	\card
	\cardstrip
	\cardbanner{banner/green.png}
	\cardicon{icons/coin.png}
	\cardprice{6}
	\cardtitle{Festplatz}
	\cardcontent{Diese Königreichkarte ist eine Punktekarte, keine Aktionskarte. Sie hat bis zum Ende des Spiels keine Funktion. Bei der Wertung zählt sie 2 Siegpunkte für je volle 5 Karten mit unterschiedlichem Namen im gesamten Kartensatz (Nachziehstapel, Ablagestapel und Handkarten) des Spielers. Bei Spielende suchst du aus deinem gesamten Kartensatz je eine Karte jedes Namens heraus. Diese Karten zählst du und teilst die Anzahl durch 5. Das Ergebnis (abgerundet) multipliziert mit 2 ergibt die Punkte. Hast du 0-4 Karten mit unterschiedlichem Namen erhältst du keine Punkte, für 5-9 Karten 2 Punkte , für 10-14 Karten 4 Punkte usw.
	
	\smallskip
	
	Im Spiel mit 2 Spielern werden 8 Karten Festplatz verwendet, bei 3 oder mehr Spielern 12 Karten.}
\end{tikzpicture}
\hspace{-0.6cm}
\begin{tikzpicture}
	\card
	\cardstrip
	\cardbanner{banner/gold.png}
	\cardicon{icons/coin.png}
	\cardprice{5}
	\cardtitle{Füllhorn}
	\cardcontent{Du spielst diese Karte, wie jede andere Geldkarte, in der Kaufphase aus. Das Füllhorn hat den Wert \coin[0]. Es bringt also kein Geld für den Kauf. Du darfst dir jedoch sofort, wenn du die Karte ausspielst, eine Karte aus dem Vorrat nehmen. Diese Karte darf bis zu \coin[1] pro Karte mit unterschiedlichem Namen, die du im Spiel hast, kosten. Karten, die du im Spiel hast sind: in diesem Zug ausgespielte Aktions- und Geldkarten (das Füllhorn selbst eingeschlossen) und bei dir ausliegende Dauerkarten (Dominion – Seaside). Nicht im Spiel sind Karten, die du nach dem ausspielen entsorgt hast (z. B. Festmahl, Dominion – Basisspiel). Nimmst du dir mit dem Füllhorn eine Punktekarte (auch eine kombinierte), entsorgst du das Füllhorn. Du entsorgst das Füllhorn nicht, wenn du dir eine Punktekarte auf eine andere Art nimmst oder kaufst.}
\end{tikzpicture}
\hspace{-0.6cm}
\begin{tikzpicture}
	\card
	\cardstrip
	\cardbanner{banner/white.png}
	\cardicon{icons/coin.png}
	\cardprice{5}
	\cardtitle{Harlekin}
	\cardcontent{Du erhältst zunächst +\coin[2]. Dann muss, beginnend mit dem Spieler links von dir, reihum jeder Mitspieler die oberste Karte von seinem Nachziehstapel ablegen. Ist es eine Punktekarte (auch eine kombinierte), so muss er sich einen Fluch nehmen. Ist kein Fluch mehr im Vorrat, nimmt der Spieler keinen. Ist es keine Punktekarte, kannst du wählen: Entweder der Spieler muss sich eine Karte mit gleichem Namen aus dem Vorrat nehmen und auf seinen Ablagestapel legen oder du nimmst dir eine Karte mit gleichem Namen und legst sie auf deinen Ablagestapel. Ist keine Karte mit diesem Namen mehr im Vorrat nimmt keiner von beiden eine solche Karte.}
\end{tikzpicture}
\hspace{-0.6cm}
\begin{tikzpicture}
	\card
	\cardstrip
	\cardbanner{banner/white.png}
	\cardicon{icons/coin.png}
	\cardprice{4}
	\cardtitle{Junge Hexe}
	\cardcontent{\tiny{\begin{Spacing}{1}
	\emph{Spielvorbereitung:} Wird die Junge Hexe im Spiel verwendet, so wird zusätzlich ein Bannstapel benötigt. (Dies gilt auch, wenn sich die Junge Hexe im Schwarzmarktstapel befindet – Promokarte: Schwarzmarkt.) Für den Bannstapel wählen die Spieler eine nicht verwendete Königreichkarte, die \coin[2] oder \coin[3] kostet, aus einer beliebigen Edition oder Erweiterung aus. Der komplette Stapel dieser Karte wird als zusätzlicher Stapel in den Vorrat gelegt. Die Platzhalterkarte Junge Hexe wird quer unter den Stapel geschoben, um diesen Stapel als Bannstapel zu kennzeichnen. In diesem Spiel werden also 11 unterschiedliche Königreichskarten verwendet. Zusätzlich zur üblichen (aufgedruckten) Funktion sind alle diese Karten Bannkarten. Der Bannstapel wird behandelt wie die anderen Königreichstapel. Er ist Teil des Vorrats, die Karten können wie üblich gekauft oder genommen werden und der Stapel wird für die Spielendebedingung berücksichtigt.
	
	\smallskip
	
	Wenn du die Junge Hexe ausspielst, ziehst du zuerst 2 Karten nach, dann musst du 2 Karten aus deiner Hand ablegen. Beginnend mit dem Spieler links von dir darf nun reihum jeder Mitspieler eine Bannkarte aus seiner Hand aufdecken. Wenn er das nicht macht, muss er sich einen Fluch vom Vorrat nehmen. Ist kein Fluch mehr im Vorrat, muss der Spieler keinen nehmen. Die Spieler dürfen wie üblich auch Reaktionskarten (z. B. Pferdehändler) aus ihrer Hand aufdecken. Dies tun die Spieler, bevor sie eine Bannkarte aufdecken. Die Anweisungen auf der Reaktionskarte werden also ausgeführt, bevor der Angriff durch die Bannkarte abgewehrt wird. Ist die Bannkarte eine Reaktionskarte, kann sie zuerst aufgedeckt werden um die übliche Reaktion auszuführen und dann (wenn der Spieler sie dann noch auf der Hand hat) nochmals in ihrer Funktion als Bannkarte.
	\end{Spacing}}}
\end{tikzpicture}
\hspace{-0.6cm}
\begin{tikzpicture}
	\card
	\cardstrip
	\cardbanner{banner/white.png}
	\cardicon{icons/coin.png}
	\cardprice{3}
	\cardtitle{Menagerie}
	\cardcontent{Zuerst erhältst du +1 Aktion. Dann deckst du deine Handkarten auf. Hast du nur Karten mit unterschiedlichem Namen auf der Hand, ziehst du 3 Karten nach. Hast du wenigstens eine Karte mit gleichem Namen auf der Hand (eine Karte doppelt), ziehst du 1 Karte nach. Hast du z. B. nur ein Kupfer und ein Silber auf der Hand, hast du nur Karten mit unterschiedlichem Namen (obwohl beide Geldkarten sind). Du darfst also 3 Karten nachziehen.}
\end{tikzpicture}
\hspace{-0.6cm}
\begin{tikzpicture}
	\card
	\cardstrip
	\cardbanner{banner/white.png}
	\cardicon{icons/coin.png}
	\cardprice{4}
	\cardtitle{Nachbau}
	\cardcontent{Zuerst entsorgst du eine Karte aus deiner Hand und nimmst dir dafür eine Karte, die genau \coin[1] mehr kostet. Dann entsorgst du eine andere Karte aus deiner Hand und nimmst dir wieder eine Karte, die genau \coin[1] mehr kostet. Du nimmst die Karten vom Vorrat und legst sie auf deinen Ablagestapel. Ist im Vorrat keine Karte, die genau 1 Geld mehr kostet als die entsorgte, nimmst du dir keine Karte.}
\end{tikzpicture}
\hspace{-0.6cm}
\begin{tikzpicture}
	\card
	\cardstrip
	\cardbanner{banner/blue.png}
	\cardicon{icons/coin.png}
	\cardprice{4}
	\cardtitle{\footnotesize{Pferdehändler}}
	\cardcontent{Wenn du diese Karte in deinem Zug ausspielst, erhältst du zuerst +1 Kauf und +\coin[3]. Dann musst du 2 Karten aus deiner Hand ablegen. Hast du weniger als 2 Karten auf der Hand, legst du nur so viele wie möglich ab.

	\smallskip
				
	Spielt ein Mitspieler eine Angriffskarte, darfst du diese Karte aus deiner Hand vorzeigen und dann zur Seite legen. Dann wird der Angriff normal ausgeführt. Du kannst jeden Pferdehändler nur gegen einen Angriff verwenden, da du ihn danach nicht mehr auf der Hand hast. Du darfst mehrere Pferdehändler gegen den selben Angriff zur Seite legen. Zu Beginn deines nächsten Zuges nimmst du alle zur Seite gelegten Pferdehändler wieder auf die Hand und ziehst zusätzlich für jeden dieser Pferdehändler 1 Karte nach.}
\end{tikzpicture}
\hspace{-0.6cm}
\begin{tikzpicture}
	\card
	\cardstrip
	\cardbanner{banner/white.png}
	\cardicon{icons/coin.png}
	\cardprice{5}
	\cardtitle{Treibjagd}
	\cardcontent{Zuerst ziehst du eine Karte nach und erhältst +1 Aktion. Dann deckst du deine gesamten Handkarten auf und legst diese offen aus. Danach deckst du solange Karten von deinem Nachziehstapel auf, bis du eine Karte aufdeckst, die du nicht auf der (momentan offen ausliegenden) Hand hast. Nimm diese gerade aufgedeckte Karte und deine offen ausliegenden Handkarten zurück auf die Hand. Lege die übrigen aufgedeckten Karten auf deinen Ablagestapel. Kannst du (auch nach dem Mischen deines Ablagestapels) keine Karte aufdecken, von der du nicht bereits eine auf der Hand hast, legst du die aufgedeckten Karten ab und nimmst nur deine Handkarten zurück auf die Hand.}
\end{tikzpicture}
\hspace{-0.6cm}
\begin{tikzpicture}
	\card
	\cardstrip
	\cardbanner{banner/white.png}
	\cardicon{icons/coin.png}
	\cardprice{4}
	\cardtitle{Turnier}
	\cardcontent{\miniscule{\begin{Spacing}{1}
	\emph{Spielvorbereitung:} Wird das Turnier für das Spiel verwendet, so wird zusätzlich der Preisstapel benötigt. (Dies gilt auch, wenn sich das Turnier im Schwarzmarktstapel befindet – Promokarte: Schwarzmarkt.) Für den Preisstapel werden die 5 einmaligen Preiskarten benötigt. Alle Preiskarten werden als ein Stapel offen bereit gelegt. Der Preisstapel besteht nur aus diesen 5 Karten und ist nicht Teil des Vorrats. 
	
	\smallskip

	\emph{Errata:} Der Kartentext is falsch, es sollte \enquote{Jeder Spieler darf eine Provinz aus seiner Hand aufdecken. Wenn du das machst, lege die Provinz ab und nimm dir einen Preis (vom Preisstapel) oder ein Herzogtum und lege die neue Karte auf deinen Nachziehstapel. Wenn es kein Mitspieler macht: +1 Karte, +\coin[1].} statt \enquote{Du darfst eine Provinz aus deiner Hand ablegen. Wenn du das machst [...] Jeder Mitspieler darf...} heißen. 
	
	\smallskip

	Zuerst erhältst du +1 Aktion. Dann darf jeder Spieler eine Provinz aus seiner Hand aufdecken. Wenn du das machst, lege die Provinz ab und nehme dir eine frei wählbare Preiskarte vom Preisstapel oder ein Herzogtum vom Vorrat und lege die neue Karte auf deinen Nachziehstapel. Du kannst dich auch für einen der beiden Stapel entscheiden, wenn dieser leer ist. Ist der Stapel für den du dich entscheidest leer, nimmst du dir keine Karte. Wenn keiner deiner Mitspieler eine Provinz aufgedeckt hat, erhältst du +1 Karte und +\coin[1]. Es gibt also 4 mögliche Ergebnisse:
	\begin{itemize}	
	\item Du legst keine Provinz ab und keiner deiner Mitspieler deckt eine Provinz auf: Du erhältst +1 Karte, +1 Aktion und +\coin[1].\\
	\item Du legst keine Provinz ab und mindestens einer deiner Mitspieler deckt eine Provinz auf: Du erhältst +1 Aktion.\\
	\item Du legst eine Provinz ab aber keiner deiner Mitspieler deckt eine Provinz auf: Du nimmst dir eine Preiskarte oder ein Herzogtum und erhältst +1 Karte, +1 Aktion und +\coin[1].\\
	\item Du legst eine Provinz ab und mindestens einer deiner Mitspieler deckt eine Provinz auf: Du nimmst dir eine Preiskarte oder ein Herzogtum und erhältst +1 Aktion.\\
	\end{itemize}
	Du darfst den Preisstapel jederzeit durchsehen.
	\end{Spacing}}}
\end{tikzpicture}
\hspace{-0.6cm}
\begin{tikzpicture}
	\card
	\cardstrip
	\cardbanner{banner/white.png}
	\cardicon{icons/coin.png}
	\cardprice{3}
	\cardtitle{Wahrsagerin}
	\cardcontent{Du erhältst zunächst +\coin[2]. Dann muss, beginnend mit dem Spieler links von dir, reihum jeder Mitspieler solange Karten von seinem Nachziehstapel aufdecken, bis entweder eine Punktekarte oder ein Fluch offen liegt. Er legt diese Punkte- oder Fluchkarte verdeckt auf seinen Nachziehstapel. Der Spieler darf nicht wählen, ob er eine Punktekarte oder einen Fluch auf den Nachziehstapel legt, er muss die erste aufgedeckte Punkte- oder Fluchkarte zurücklegen. Die übrigen aufgedeckten Karten legt er auf seinen Ablagestapel. Wenn der Spieler auch nach dem Mischen seines Ablagestapels keine Punkte- oder Fluchkarte aufdecken kann, legt er alle aufgedeckten Karten ab. Karten mit kombinierten Kartentypen sind gleichzeitig alle angegebenen Kartentypen.}
\end{tikzpicture}
\hspace{-0.6cm}
\begin{tikzpicture}
	\card
	\cardstrip
	\cardbanner{banner/white.png}
	\cardicon{icons/coin.png}
	\cardprice{2}
	\cardtitle{Weiler}
	\cardcontent{Zuerst ziehst du immer eine Karte nach und erhältst +1 Aktion. Nun darfst du eine Karte aus deiner Hand ablegen um entweder zusätzlich +1 Aktion oder +1 Kauf zu erhalten. Oder du legst 2 Karten aus deiner Hand ab und erhältst zusätzlich +1 Aktion und +1 Kauf. Du kannst jedoch nicht +2 Aktionen oder +2 Käufe zusätzlich erhalten. Du kannst auch darauf verzichten, Karten abzulegen und erhältst dann nichts weiter.}
\end{tikzpicture}
\hspace{-0.6cm}
\begin{tikzpicture}
	\card
	\cardstrip
	\cardbanner{banner/white.png}
	\cardicon{icons/coin.png}
	\cardprice{0*}
	\cardtitle{Preiskarten}
	\cardcontent{\tiny{\begin{Spacing}{1}	
	\emph{Diadem:} Diese Geldkarte hat den Wert 2, wie ein Silber. Wenn du sie in deiner Kaufphase auslegst, erhältst du zusätzlich +1 virtuelles Geld für jede unverbrauchte Aktion. Hast du z. B. in deiner Aktionsphase keine Aktionskarte ausgespielt, so erhältst du für deine freie Aktion +1 Geld. Hast du in deiner Aktionsphase das Bauerndorf ausgespielt, so ist dadurch deine freie Aktion verbraucht. Das Bauerndorf gibt dir jedoch +2 Aktionen. Spielst du keine weitere Aktionskarte mehr aus, bringt dir das Diadem in der Kaufphase +2 Geld.

	\medskip

	\emph{Ein Sack voll Gold:} Zuerst erhältst du +1 Aktion. Dann nimmst du dir ein Gold aus dem Vorrat und legst es sofort auf deinen Nachziehstapel. Falls der Nachziehstapel leer ist, legst du dieses Gold an die Stelle deines Nachziehstapels. Ist kein Gold mehr im Vorrat, nimmst du keines.

	\medskip

	\emph{Gefolge:} Du ziehst zunächst 2 Karten von deinem Nachziehstapel. Dann nimmst du ein Anwesen vom Vorrat und legst es auf deinen Ablagestapel. Ist kein Anwesen mehr im Vorrat, nimmst du dir keines. Nun muss sich, beginnend mit dem Spieler links von dir, reihum jeder Mitspieler einen Fluch vom Vorrat nehmen und auf seinen Ablagestapel legen und danach zusätzlich solange Karten aus seiner Hand ablegen, bis er nur noch 3 Karten auf der Hand hält. Hat ein Mitspieler 3 oder weniger Karten auf der Hand, muss er keine Karten ablegen (er muss sich jedoch trotzdem einen Fluch nehmen). Ist kein Fluch mehr im Vorrat, nimmt der Spieler sich keinen.
	\end{Spacing}}}
\end{tikzpicture}
\hspace{-0.6cm}
\begin{tikzpicture}
	\card
	\cardstrip
	\cardbanner{banner/white.png}
	\cardicon{icons/coin.png}
	\cardprice{0*}
	\cardtitle{Preiskarten}
	\cardcontent{\tiny{\begin{Spacing}{1}
	\emph{Prinzessin:} Zuerst erhältst du +1 Kauf. Wenn diese Karte im Spiel ist, kosten alle Karten um 2 Geld weniger, niemals jedoch weniger als 0 Geld. Dies betrifft alle Karten im Vorrat, in den Nachzieh- und den Ablagestapeln aller Spieler und auch die Handkarten der Spieler. Wenn du mit dem Nachbau ein Kupfer (kostet weiterhin 0 Geld) aus deiner Hand entsorgst, kannst du dir dafür z. B. ein Silber nehmen, das nur noch 1 Geld kostet. Spielst du die Prinzessin auf einen Thronsaal, sind die Karten trotzdem nur um 2 Geld billiger, da die Prinzessin nur einmal im Spiel ist.

	\medskip

	\emph{Errata:} Der Kartentext des Streitross ist falsch, es sollte \enquote{[...] und lege sofort deinen kompletten Nachziehstapel auf deinen Ablagestapel.} statt \enquote{[...] und lege sofort deinen kompletten Nachziehstapel ab.} heißen. Durch das direkte Ablegen wird die Karte Tunnel (Dominion – Hinterland) nicht ausgelöst.

	\medskip

	\emph{Streitross:} Zuerst wählst du 2 unterschiedliche der 4 Anweisungen auf der Karte. Dann führst du die beiden gewählten Anweisungen in der Reihenfolge auf der Karte aus. Wenn du z. B. +2 Karten und die 4 Silber wählst, ziehst du die 2 Karten nach, bevor du die 4 Silber nimmst und deinen Nachziehstapel auf den Ablagestapel legst. Sind im Vorrat weniger als 4 Silber nimmst du dir nur die übrigen Silber im Vorrat. Du darfst deinen Nachziehstapel nicht durchsehen, bevor du ihn auf den Ablagestapel legst.
	\end{Spacing}}}
\end{tikzpicture}
\hspace{-0.6cm}
\begin{tikzpicture}
	\card
	\cardstrip
	\cardbanner{banner/white.png}
	\cardtitle{\scriptsize{Empfohlene 10er Sätze\qquad}}
	\cardcontent{\emph{Reiche Ernte	und	Basisspiel:}

	\smallskip

	\emph{Kopfgeld:}\\
	Ernte, Füllhorn, Menagerie, Treibjagd, Turnier (+ Preiskarten) / Geldverleiher, Jahrmarkt, Keller, Miliz, Schmiede

	\smallskip

	\emph{Böses Omen:}\\
	Füllhorn, Harlekin, Nachbau, Wahrsagerin, Weiler / Abenteurer, Bürokrat, Laboratorium, Spion, Thronsaal

	\smallskip

	\emph{Wanderzirkus:}\\
	Bauerndorf, Festplatz, Harlekin, Junge Hexe (+ Kanzler als Bannstapel), Pferdehändler / Festmahl, Laboratorium, Markt, Umbau, Werkstatt}
\end{tikzpicture}
\hspace{-0.6cm}
\begin{tikzpicture}
	\card
	\cardstrip
	\cardbanner{banner/white.png}
	\cardtitle{\scriptsize{Empfohlene 10er Sätze\qquad}}
	\cardcontent{\emph{Reiche Ernte und	Die	Intrige}

	\smallskip

	\emph{Wer zuletzt lacht:}\\
	Bauerndorf, Ernte, Harlekin, Pferdehändler, Treibjagd / Adlige, Handlanger, Lakai, Trickser, Verwalter

	\smallskip

	\emph{Würze des Lebens:}\\
	Festplatz, Füllhorn, Junge Hexe (+ Wunschbrunnen als Bannstapel), Nachbau, Turnier (+ Preiskarten) / 	Bergwerk, Burghof, Große Halle, Kupferschmied, Tribut 

	\smallskip

	\emph{Kleine  Siege:}\\
	Nachbau, Treibjagd, Turnier (+ Preiskarten), Wahrsagerin, Weiler / Große Halle, Handlanger, Harem, Herzog, Verschwörer}
\end{tikzpicture}
\hspace{-0.6cm}
\begin{tikzpicture}
	\card
	\cardstrip
	\cardbanner{banner/white.png}
	\cardtitle{Platzhalter\quad}
\end{tikzpicture}
\hspace{0.6cm}
