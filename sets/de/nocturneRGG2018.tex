% Basic settings for this card set
\renewcommand{\cardcolor}{nocturne}
\renewcommand{\cardextension}{Erweiterung X}
\renewcommand{\cardextensiontitle}{Nocturne}
\renewcommand{\seticon}{nocturne.png}

\clearpage
\newpage
\section{\cardextension \ - \cardextensiontitle \ (Rio Grande Games 2018)}

\begin{tikzpicture}
	\card
	\cardstrip
	\cardbanner{banner/white.png}
	\cardicon{icons/coin.png}
	\cardprice{2}
	\cardtitle{Druidin}
	\cardcontent{In der Spielvorbereitung werden die obersten 3 \emph{Gaben} aufgedeckt neben dem \emph{Druidinnen}-Stapel zur Seite gelegt. Diese 3 \emph{Gaben} werden in diesem Spiel \emph{ausschließlich} für die \emph{DRUIDIN} verwendet. Verwendet ihr weitere SEGEN-Karten im Spiel, besteht der entsprechende \emph{Gaben}-Stapel aus den restlichen neun \emph{Gaben}.
		
	\medskip
		
	Wenn du die \emph{DRUIDIN} ausspielst, wähle eine der drei zur Seite gelegten \emph{Gaben}, empfange sofort die entsprechende \emph{Gabe}, lasse die \emph{Gabe} aber zur Seite gelegt (auch wenn die \emph{Gabe} eigentlich sagt, du sollst sie bis zu deiner Aufräumphase aufbewahren).}
\end{tikzpicture}
\hspace{-0.6cm}
\begin{tikzpicture}
	\card
	\cardstrip
	\cardbanner{banner/white.png}
	\cardicon{icons/coin.png}
	\cardprice{2}
	\cardtitle{\footnotesize{Fährtensucher}}
	\cardcontent{In der Spielvorbereitung erhält jeder Spieler ein ERBSTÜCK \emph{BEUTEL} und dafür ein \emph{KUPFER} weniger.
		
	\medskip
		
	Wenn du diese Karte ausspielst und danach mehr als eine Karte nimmst (kaufst oder auf andere Art und Weise nimmst), darfst du für jede genommene Karte neu entscheiden, ob du sie auf deinen Nachziehstapel legst. Wenn du die durch den \emph{FÄHRTENSUCHER} empfangene \emph{Gabe} ausführst und dadurch eine Karte nimmst (z.B. ein \emph{SILBER} durch \emph{GESCHENK DES BERGES}), darfst du diese auf deinen Nachziehstapel legen.}
\end{tikzpicture}
\hspace{-0.6cm}
\begin{tikzpicture}
	\card
	\cardstrip
	\cardbanner{banner/white.png}
	\cardicon{icons/coin.png}
	\cardprice{2}
	\cardtitle{Fee}
	\cardcontent{In der Spielvorbereitung erhält jeder Spieler ein ERBSTÜCK \emph{ZIEGE} und dafür ein \emph{KUPFER} weniger.
		
	\medskip
		
	Wenn du die \emph{FEE} nicht entsorgst, erhältst du die \emph{Gabe} gar nicht. Wenn du die abgelegte \emph{Gabe} laut Anweisung bis zu deiner Aufräumphase behalten sollst, lege sie vor dir ab, merke dir, dass du sie zweimal empfängst und lege sie in deiner Aufräumphase ab.}
\end{tikzpicture}
\hspace{-0.6cm}
\begin{tikzpicture}
	\card
	\cardstrip
	\cardbanner{banner/blue.png}
	\cardicon{icons/coin.png}
	\cardprice{2}
	\cardtitle{\footnotesize{Getreuer Hund}}
	\cardcontent{Die Reaktion kann entweder im eigenen Zug (wenn die Karte zu einem anderen Zeitpunkt außer der Aufräumphase abgelegt wird) oder während des Zugs eines Mitspielers, wenn z.B. durch einen Angriff die Karte abgelegt werden muss, zum Tragen kommen. Wenn du die Reaktion nutzen möchtest, lege diesen \emph{GETREUEN HUND} zur Seite, anstatt ihn abzulegen und nimm ihn am Ende des Zuges wieder auf die Hand (wenn es dein eigener Zug war, nach dem Ziehen der neuen Kartenhand).
		
	\medskip

	Um die Reaktion des \emph{GETREUEN HUNDES} zu nutzen, muss er sich nicht zwingend auf der Hand befinden. Muss er zum Beispiel auf Grund der \emph{NACHTWACHE} direkt aus dem Nachziehstapel abgelegt werden, darfst du die Reaktion nutzen. Musst du den \emph{GETREUEN HUND} auf den Ablagestapel legen, ohne ihn im spieltechnischen Sinn abzulegen (z.B. nach dem Kauf oder durch \emph{LUMPENSAMMLER} aus \emph{Dark Ages}), passiert nichts. Du kannst den \emph{GETREUEN HUND} aber nur außerhalb der Aufräumphase ablegen (bzw. zur Seite legen), wenn du durch eine Anweisung dazu aufgefordert wirst.}
\end{tikzpicture}
\hspace{-0.6cm}
\begin{tikzpicture}
	\card
	\cardstrip
	\cardbanner{banner/black.png}
	\cardicon{icons/coin.png}
	\cardprice{2}
	\cardtitle{Kloster}
	\cardcontent{Diese Karte ist eine Nachtkarte und kann nur in der Nachtphase ausgespielt werden. Wenn du zum Beispiel bis zu dem Zeitpunkt, an dem du das \emph{KLOSTER} ausspielst, drei Karten genommen hast, darfst du 0 bis 3 Karten entsorgen – du darfst Handkarten und/oder \emph{KUPFER}, die sich gerade im Spiel befinden, entsorgen, in jeder beliebigen Kombination. Eingetauschte Karten (z.B. eine \emph{VAMPIRIN}, die für eine \emph{FLEDERMAUS} eingetauscht wurde) zählen nicht als \enquote{genommen}.}
\end{tikzpicture}
\hspace{-0.6cm}
\begin{tikzpicture}
	\card
	\cardstrip
	\cardbanner{banner/orangeblack.png}
	\cardicon{icons/coin.png}
	\cardprice{2}
	\cardtitle{Wächterin}
	\cardcontent{Diese Karte ist eine Nachtkarte und kann nur in der Nachtphase ausgespielt werden. Wenn du diese Karte nimmst, nimm sie direkt auf die Hand, anstatt sie auf deinen Ablagestapel zu legen. Da die Nachtphase nach der Kaufphase kommt, kannst du diese Karte in dem Zug ausspielen, in der du sie gekauft oder auf andere Weise vorher im Zug genommen hast.
		
	\medskip
		
	Solange sich diese Karte im Spiel befindet, bist du von allen ausgespielten Angriffskarten deiner Mitspieler nicht betroffen (auch nicht, wenn du das möchtest). Lege die Karte in der Aufräumphase deines nächsten Zuges ab.}
\end{tikzpicture}
\hspace{-0.6cm}
\begin{tikzpicture}
	\card
	\cardstrip
	\cardbanner{banner/orange.png}
	\cardicon{icons/coin.png}
	\cardprice{3}
	\cardtitle{\footnotesize{Geheime Höhle}}
	\cardcontent{In der Spielvorbereitung erhält jeder Spieler ein ERBSTÜCK \emph{WUNDERLAMPE} und dafür ein \emph{KUPFER} weniger.
		
	\medskip
		
	Wenn du nicht genau 3 Karten ablegst, wird die \emph{GEHEIME HÖHLE} in der Aufräumphase abgelegt. Wenn du genau 3 Karten ablegst, bleibt die \emph{GEHEIME HÖHLE} bis zum Ende des nächsten Zuges im Spiel und du erhältst zu Beginn des nächsten Zuges +\coin[3] . Du kannst wählen, 3 Karten abzulegen, auch wenn du weniger als 3 Karten auf der Hand hast und alle deine Handkarten ablegen – den Bonus erhältst du jedoch nicht und du legst die \emph{GEHEIME HÖHLE} am Ende des Zuges ab. Hast du mehr als 3 Karten auf der Hand, musst du entweder genau 3 Karten oder gar keine ablegen.}
\end{tikzpicture}
\hspace{-0.6cm}
\begin{tikzpicture}
	\card
	\cardstrip
	\cardbanner{banner/orangeblack.png}
	\cardicon{icons/coin.png}
	\cardprice{3}
	\cardtitle{Geisterstadt}
	\cardcontent{Diese Karte ist eine Nachtkarte und kann nur in der Nachtphase ausgespielt werden. Wenn du diese Karte nimmst, nimm sie direkt auf die Hand, anstatt sie auf den Ablagestapel zu legen. Da die Nachtphase nach der Kaufphase kommt, kannst du diese Karte in dem Zug ausspielen, in der du sie gekauft oder auf andere Weise vorher im Zug genommen hast.}
\end{tikzpicture}
\hspace{-0.6cm}
\begin{tikzpicture}
	\card
	\cardstrip
	\cardbanner{banner/white.png}
	\cardicon{icons/coin.png}
	\cardprice{3}
	\cardtitle{Kobold}
	\cardcontent{Im Spiel befinden sich der ausgespielte \emph{KOBOLD} selbst, andere in diesem Zug ausgespielte Karten, Dauerkarten aus vorherigen Zügen sowie aufgerufene Karten (aus \emph{Abenteuer}). Nicht im Spiel befinden sich bereits entsorgte sowie zur Seite gelegte Karten.}
\end{tikzpicture}
\hspace{-0.6cm}
\begin{tikzpicture}
	\card
	\cardstrip
	\cardbanner{banner/black.png}
	\cardicon{icons/coin.png}
	\cardprice{3}
	\cardtitle{Nachtwache}
	\cardcontent{Diese Karte ist eine Nachtkarte und kann nur in der Nachtphase ausgespielt werden. Wenn du diese Karte nimmst, nimm sie direkt auf die Hand, anstatt sie auf den Ablagestapel zu legen. Da die Nachtphase nach der Kaufphase kommt, kannst du diese Karte in dem Zug ausspielen, in der du sie gekauft oder auf andere Weise vorher im Zug genommen hast.}
\end{tikzpicture}
\hspace{-0.6cm}
\begin{tikzpicture}
	\card
	\cardstrip
	\cardbanner{banner/white.png}
	\cardicon{icons/coin.png}
	\cardprice{3}
	\cardtitle{Narr}
	\cardcontent{In der Spielvorbereitung erhält jeder Spieler ein ERBSTÜCK \emph{GLÜCKSTALER} und dafür ein \emph{KUPFER} weniger.
		
	\medskip
		
	Wenn du bereits den \emph{Zustand} \emph{IM WALD VERIRRT} vor dir liegen hast, passiert nichts. Wenn du \emph{IM WALD VERIRRT} nicht hast, erhalte es (wenn es gerade bei einem Spieler liegt, gibt er es dir), lege es vor dir ab und decke dann die obersten 3 \emph{Gaben} des \emph{Gaben}-Stapels auf. Empfange die \emph{Gaben} in einer von dir festgelegten Reihenfolge (diese musst du nicht zu Beginn festlegen – du kannst eine \emph{Gabe} empfangen und dann die nächste wählen usw.) und lege sie direkt nach Empfang ab bzw. bewahre sie bis zu deiner Aufräumphase auf, wenn eine \emph{Gabe} dies erfordert. Der \emph{Zustand} bleibt vor dir liegen, bis ein anderer Spieler ihn mit Hilfe des \emph{NARREN} erhält.}
\end{tikzpicture}
\hspace{-0.6cm}
\begin{tikzpicture}
	\card
	\cardstrip
	\cardbanner{banner/black.png}
	\cardicon{icons/coin.png}
	\cardprice{3}
	\cardtitle{Wechselbalg}
	\cardcontent{Immer wenn ihr mit der Königreichkarte \emph{WECHSELBALG} spielt und sich noch \emph{WECHSELBALG}-Karten im Vorrat befinden, darfst du, wenn du eine Karte nimmst, die in diesem Moment mindestens \coin[3] kostet, die genommene Karte in einen \emph{WECHSELBALG} eintauschen. Lege die genommene Karte zurück auf den entsprechenden Stapel (Anweisungen, die beim Nehmen der Karte eintreten, treten noch ein), nimm einen \emph{WECHSELBALG} und lege ihn auf den Ablagestapel. Karten, die gar kein \coin oder weniger als \coin[3] sowie z.B. \potion (aus \emph{Alchemisten}) oder \hex (aus \emph{Empires}) kosten, dürfen nicht getauscht werden, da sie niemals mehr als \coin[3] kosten, egal wie hoch die zusätzlichen Kosten in Form von \potion oder \hex sind. So darf z. B. eine \emph{VERWANDLUNG} (aus \emph{Alchemisten}) nicht getauscht werden, da sie nur \potion aber kein \coin kostet. Der \emph{ALCHEMIST} (aus \emph{Alchemisten}) hingegen darf eingetauscht werden, da dieser \coin[3] \potion kostet.
		
	\medskip
		
	Diese Karte ist eine Nachtkarte und kann nur in der Nachtphase ausgespielt werden. Wenn du das tust, entsorge diesen \emph{WECHSELBALG} und nimm dir für eine Karte, die du im Spiel hast (das können Aktions-, Geld- und/oder Nachtkarten sein), eine gleiche Karte vom entsprechenden Stapel.}
\end{tikzpicture}
\hspace{-0.6cm}
\begin{tikzpicture}
	\card
	\cardstrip
	\cardbanner{banner/white.png}
	\cardicon{icons/coin.png}
	\cardprice{4}
	\cardtitle{Attentäter}
	\cardcontent{Wenn du diese Karte ausspielst, decke die oberste \emph{Plage} auf. Jeder Mitspieler führt die Anweisung darauf (im Uhrzeigersinn, beginnend bei deinem linken Nachbarn) aus. Lege die \emph{Plage} danach auf den separaten \emph{Plagen}-Ablagestapel.}
\end{tikzpicture}
\hspace{-0.6cm}
\begin{tikzpicture}
	\card
	\cardstrip
	\cardbanner{banner/black.png}
	\cardicon{icons/coin.png}
	\cardprice{4}
	\cardtitle{Exorzistin}
	\cardcontent{Diese Karte ist eine Nachtkarte und kann nur in der Nachtphase ausgespielt werden. Wenn du das tust, entsorgst du eine beliebige Handkarte. Kostet die entsorgte Karte mehr als eine oder mehrere der ERSCHEINUNGEN (\emph{IRRLICHT} \coin[0] , \emph{TEUFELCHEN} \coin[2] , \emph{GEIST} \coin[4] ), nimmst du eine der billigeren ERSCHEINUNGEN und legst sie auf deinen Ablagestapel. Du darfst auch eine Karte entsorgen, die nicht weniger als eine der ERSCHEINUNGEN kostet (z.B. einen \emph{FLUCH} oder ein \emph{KUPFER}), nimmst dir dafür aber keine ERSCHEINUNG.}
\end{tikzpicture}
\hspace{-0.6cm}
\begin{tikzpicture}
	\card
	\cardstrip
	\cardbanner{banner/green.png}
	\cardicon{icons/coin.png}
	\cardprice{4}
	\cardtitle{Friedhof}
	\cardcontent{In der Spielvorbereitung erhält jeder Spieler ein ERBSTÜCK \emph{ZAUBERSPIEGEL} und dafür ein \emph{KUPFER} weniger.
		
	\medskip
		
	Wenn du einen \emph{FRIEDHOF} nimmst, entsorge 0 bis 4 Karten aus deiner Hand.}
\end{tikzpicture}
\hspace{-0.6cm}
\begin{tikzpicture}
	\card
	\cardstrip
	\cardbanner{banner/white.png}
	\cardicon{icons/coin.png}
	\cardprice{4}
	\cardtitle{Konklave}
	\cardcontent{Du darfst eine Aktionskarte aus deiner Hand ausspielen, von der du gerade keine gleiche im Spiel hast. Du kannst eine Karte ausspielen, die du in diesem Zug gespielt, aber bereits entsorgt hast. Dauerkarten aus vergangenen Zügen befinden sich im Spiel und dürfen entsprechend nicht ausgespielt werden. Nicht ausgespielt werden darf eine weitere \emph{KONKLAVE}, da eine solche sich bereits im Spiel befindet. Wenn du eine regelgerechte Aktionskarte gespielt hast, erhältst du eine weitere Aktion – diese hat keine Limitationen, du kannst damit jede beliebige Aktionskarte aus deiner Hand ausspielen.}
\end{tikzpicture}
\hspace{-0.6cm}
\begin{tikzpicture}
	\card
	\cardstrip
	\cardbanner{banner/white.png}
	\cardicon{icons/coin.png}
	\cardprice{4}
	\cardtitle{\footnotesize{Minnesängerin}}
	\cardcontent{Um die \emph{Gabe} zu empfangen, decke die oberste \emph{Gabe} des \emph{Gaben}-Stapels auf, empfange die \emph{Gabe} und lege sie (außer die \emph{Gabe} sagt dir etwas anderes) auf den separaten \emph{Gaben}-Ablagestapel.}
\end{tikzpicture}
\hspace{-0.6cm}
\begin{tikzpicture}
	\card
	\cardstrip
	\cardbanner{banner/white.png}
	\cardicon{icons/coin.png}
	\cardprice{4}
	\cardtitle{Schäferin}
	\cardcontent{In der Spielvorbereitung erhält jeder Spieler ein ERBSTÜCK \emph{WEIDELAND} und dafür ein \emph{KUPFER} weniger.
		
	\medskip
		
	Wenn du zum Beispiel 3 Punktekarten (auch ggf. kombinierte) ablegst, ziehst du 6 Karten nach.}
\end{tikzpicture}
\hspace{-0.6cm}
\begin{tikzpicture}
	\card
	\cardstrip
	\cardbanner{banner/white.png}
	\cardicon{icons/coin.png}
	\cardprice{4}
	\cardtitle{Seliges Dorf}
	\cardcontent{Decke die \emph{Gabe} auf, bevor du dich entscheidest, ob du sie sofort oder erst zu Beginn deines nächsten Zuges empfangen möchtest. Wenn du sie erst in deinem nächsten Zug empfangen möchtest, lege sie vor dir ab und lege sie nach Empfang (oder in der Aufräumphase nach Empfang, wenn die \emph{Gabe} dies anweist) auf den separaten \emph{Gaben}-Ablagestapel.}
\end{tikzpicture}
\hspace{-0.6cm}
\begin{tikzpicture}
	\card
	\cardstrip
	\cardbanner{banner/black.png}
	\cardicon{icons/coin.png}
	\cardprice{4}
	\cardtitle{\scriptsize{Teufelswerkstatt}}
	\cardcontent{Diese Karte ist eine Nachtkarte und kann nur in der Nachtphase ausgespielt werden. Es zählen alle Karten, die bisher in diesem Zug genommen wurden, auch in der Nachtphase bis zum Ausspielen dieser \emph{TEUFELSWERKSTATT}. Du kannst dich nicht entscheiden, einen anderen Bonus zu nehmen – wenn du 2 oder mehr Karten genommen hast, musst du ein \emph{TEUFELCHEN} nehmen. Du darfst nicht stattdessen ein \emph{GOLD} oder eine Karte, die bis zu \coin[4] kostet, nehmen. Eingetauschte Karten (z.B. \emph{VAMPIRIN} für eine \emph{FLEDERMAUS}) gelten nicht als \enquote{genommen}.}
\end{tikzpicture}
\hspace{-0.6cm}
\begin{tikzpicture}
	\card
	\cardstrip
	\cardbanner{banner/white.png}
	\cardicon{icons/coin.png}
	\cardprice{4}
	\cardtitle{\scriptsize{Totenbeschwörer}}
	\cardcontent{In der Spielvorbereitung werden die 3 ZOMBIES aufgedeckt auf den Müllstapel gelegt. Der \emph{TOTENBESCHWÖRER} kann damit mindestens einen der 3 ZOMBIES spielen, da diese von Spielbeginn an im Müll liegen. Im Verlauf des Spiels können weitere Aktionskarten, die im Müll landen, gespielt werden.
		
	\medskip
		
	Spiele eine Aktionskarte, die mit der Vorderseite nach oben im Müll liegt und keine Dauerkarte ist, lasse sie dort und drehe sie für diesen Zug mit der Bildseite nach unten – damit kann jede Karte des Müllstapels maximal einmal pro Zug gespielt werden. Am Ende des Zuges wird die Karte wieder umgedreht. Die so gespielte Aktionskarte befindet sich nicht \enquote{im Spiel} und verbleibt auf jeden Fall im Müllstapel, auch wenn sie durch die Anweisung eigentlich anderswo hingelegt werden würde.}
\end{tikzpicture}
\hspace{-0.6cm}
\begin{tikzpicture}
	\card
	\cardstrip
	\cardbanner{banner/white.png}
	\cardicon{icons/coin.png}
	\cardprice{5}
	\cardtitle{Folterknecht}
	\cardcontent{Du musst das \emph{TEUFELCHEN} nehmen, wenn du keine anderen Karten im Spiel hast. Wenn du andere Karten im Spiel hast, decke die oberste \emph{Plage} auf. Jeder Mitspieler führt die Anweisung darauf (im Uhrzeigersinn, beginnend bei deinem linken Nachbarn) aus. Lege die \emph{Plage} danach auf den separaten \emph{Plagen}-Ablagestapel.}
\end{tikzpicture}
\hspace{-0.6cm}
\begin{tikzpicture}
	\card
	\cardstrip
	\cardbanner{banner/gold.png}
	\cardicon{icons/coin.png}
	\cardprice{5}
	\cardtitle{Götze}
	\cardcontent{Wichtig ist, wie viele \emph{GÖTZEN} du in diesem Moment im Spiel hast, nicht wie viele du in diesem Zug gespielt hat (es gibt Karten, z.B. \emph{FALSCHGELD} aus \emph{Dark Ages}, durch die diese Zahl unterschiedlich sein kann).}
\end{tikzpicture}
\hspace{-0.6cm}
\begin{tikzpicture}
	\card
	\cardstrip
	\cardbanner{banner/white.png}
	\cardicon{icons/coin.png}
	\cardprice{5}
	\cardtitle{Heiliger Hain}
	\cardcontent{Du musst die \emph{Gabe} empfangen. Mit Ausnahme von \emph{GESCHENK DES FELDES} und \emph{GESCHENK DES WALDES}, die +\coin[1] geben, können alle Mitspieler wählen, ob sie die \emph{Gabe} ebenfalls empfangen wollen oder nicht. Bei \emph{GESCHENK DES FLUSSES} zieht jeder Mitspieler, der sich für die \emph{Gabe} entscheidet, am Ende \emph{deines} Zuges eine FELDES und GESCHENK DES WALDES, die + Karte.}
\end{tikzpicture}
\hspace{-0.6cm}
\begin{tikzpicture}
	\card
	\cardstrip
	\cardbanner{banner/orangeblack.png}
	\cardicon{icons/coin.png}
	\cardprice{5}
	\cardtitle{Krypta}
	\cardcontent{Diese Karte ist eine Nachtkarte und kann nur in der Nachtphase ausgespielt werden. Sie bleibt bis zur Aufräumphase des Zuges im Spiel, in dem die letzte zur Seite gelegte Geldkarte auf die Hand genommen wird. Lege die Geldkarten verdeckt unter die \emph{KRYPTA}. Du darfst dir die Karten jederzeit anschauen, deine Mitspieler aber nicht. Du darfst dich auch entscheiden, keine Karte zur Seite zu legen – dann legst du die \emph{KRYPTA} am Ende des Zuges ab.}
\end{tikzpicture}
\hspace{-0.6cm}
\begin{tikzpicture}
	\card
	\cardstrip
	\cardbanner{banner/white.png}
	\cardicon{icons/coin.png}
	\cardprice{5}
	\cardtitle{Puka}
	\cardcontent{In der Spielvorbereitung erhält jeder Spieler ein ERBSTÜCK \emph{VERFLUCHTES GOLD} und dafür ein \emph{KUPFER} weniger.}
\end{tikzpicture}
\hspace{-0.6cm}
\begin{tikzpicture}
	\card
	\cardstrip
	\cardbanner{banner/orangeblack.png}
	\cardicon{icons/coin.png}
	\cardprice{5}
	\cardtitle{Schuster}
	\cardcontent{Diese Karte ist eine Nachtkarte und kann nur in der Nachtphase ausgespielt werden. Außerdem ist sie eine Dauerkarte und wird erst in der Aufräumphase deines nächsten Zuges abgelegt.}
\end{tikzpicture}
\hspace{-0.6cm}
\begin{tikzpicture}
	\card
	\cardstrip
	\cardbanner{banner/orangeblack.png}
	\cardicon{icons/coin.png}
	\cardprice{5}
	\cardtitle{Sündenpfuhl}
	\cardcontent{Diese Karte ist eine Nachtkarte und kann nur in der Nachtphase ausgespielt werden. Wenn du diese Karte nimmst, nimm sie direkt auf die Hand, anstatt sie auf den Ablagestapel zu legen. Da die Nachtphase nach der Kaufphase kommt, kannst du diese Karte in dem Zug ausspielen, in der du sie gekauft oder auf andere Weise vorher im Zug genommen hast.}
\end{tikzpicture}
\hspace{-0.6cm}
\begin{tikzpicture}
	\card
	\cardstrip
	\cardbanner{banner/white.png}
	\cardicon{icons/coin.png}
	\cardprice{5}
	\cardtitle{\scriptsize{Tragischer Held}}
	\cardcontent{Wenn du nach dem Ziehen der + 3 Karten 8 oder mehr Karten auf der Hand hast und diesen \emph{TRAGISCHEN HELDEN} entsorgen musst, erhältst du trotzdem +1 Kauf. Kannst du den \emph{TRAGISCHEN HELDEN} nicht entsorgen (weil du ihn zum Beispiel mit Hilfe des \emph{THRONSAALS} (aus dem \emph{Basisspiel}) zweimal gespielt und bereits entsorgt hast), nimmst du trotzdem eine Geldkarte.}
\end{tikzpicture}
\hspace{-0.6cm}
\begin{tikzpicture}
	\card
	\cardstrip
	\cardbanner{banner/black.png}
	\cardicon{icons/coin.png}
	\cardprice{5}
	\cardtitle{Vampirin}
	\cardcontent{Diese Karte ist eine Nachtkarte und kann nur in der Nachtphase ausgespielt werden.
		
	\medskip
		
	Führe die drei Anweisungen in der vorgegebenen Reihenfolge aus. Decke zuerst die oberste \emph{Plage} auf. Jeder Mitspieler führt die Anweisung darauf (im Uhrzeigersinn, beginnend bei deinem linken Nachbarn) aus. Lege die \emph{Plage} danach auf den separaten \emph{Plagen}-Ablagestapel. Nimm dir dann eine Karte, die bis zu 5 kostet (außer einer VAMPIRIN) und tausche dann die \emph{VAMPIRIN} in eine \emph{FLEDERMAUS} ein.}
\end{tikzpicture}
\hspace{-0.6cm}
\begin{tikzpicture}
	\card
	\cardstrip
	\cardbanner{banner/white.png}
	\cardicon{icons/coin.png}
	\cardprice{5}
	\cardtitle{\scriptsize{Verfluchtes Dorf}}
	\cardcontent{Wenn du bereits 6 oder mehr Handkarten hast, ziehe keine Karten. Wenn du das \emph{VERFLUCHTE DORF} nimmst, empfängst du eine \emph{Plage}; da dies oft in deiner Kaufphase der Fall ist, haben einige der \emph{Plagen} keine Auswirkung auf dich.}
\end{tikzpicture}
\hspace{-0.6cm}
\begin{tikzpicture}
	\card
	\cardstrip
	\cardbanner{banner/whiteblack.png}
	\cardicon{icons/coin.png}
	\cardprice{5}
	\cardtitle{Werwolf}
	\cardcontent{Diese Karte ist sowohl eine Aktionsals auch eine Nachtkarte und kann entsprechend in der Aktions- oder der Nachtphase ausgespielt werden. Spielst du den \emph{WERWOLF} in der Aktionsphase, ziehst du 3 Karten; spielst du ihn in der Nachtphase, empfängt jeder Mitspieler die nächste \emph{Plage}. Decke die oberste \emph{Plage} auf. Jeder Mitspieler führt die Anweisung darauf (im Uhrzeigersinn, beginnend bei deinem linken Nachbarn) aus. Lege die \emph{Plage} danach auf den separaten \emph{Plagen}-Ablagestapel.}
\end{tikzpicture}
\hspace{-0.6cm}
\begin{tikzpicture}
	\card
	\cardstrip
	\cardbanner{banner/orangeblack.png}
	\cardicon{icons/coin.png}
	\cardprice{6}
	\cardtitle{Plünderer}
	\cardcontent{Diese Karte ist eine Nachtkarte und kann nur in der Nachtphase ausgespielt werden. Mitspieler können auf das Ausspielen dieser Angriffskarte mit einer entsprechenden Reaktionskarte reagieren. Hast du zum Beispiel 3 \emph{KUPFER}, 1 \emph{SILBER} und 1 \emph{PLÜNDERER} im Spiel, müssen alle Mitspieler mit 5 oder mehr Handkarten ein \emph{KUPFER}, ein \emph{SILBER} oder einen \emph{PLÜNDERER} (nach ihrer eigenen Wahl) ablegen oder ihre Kartenhand vorzeigen, wenn sie keine dieser Karten auf der Hand haben.}
\end{tikzpicture}
\hspace{-0.6cm}
\begin{tikzpicture}
	\card
	\cardstrip
	\cardbanner{banner/gold.png}
	\cardtitle{Gaben\qquad}
	\cardcontent{Alle 12 \emph{Gaben} sind jeweils nur 1 x im Spiel enthalten. Sie dürfen nur erhalten bzw. empfangen werden, wenn eine Königreichkarte mit dem Typ SEGEN dies anweist.
		
	\smallskip
		
	Die Anweisung einer \emph{Gabe} wird erst ausgeführt, sobald sie \emph{empfangen} wird. Nach dem Empfang der \emph{Gabe} wird diese auf einen separaten \emph{Gaben}-Ablagestapel gelegt. \emph{Gaben} werden niemals Bestandteil des Kartensatzes eines Spielers.
	
	\medskip
	
	\emph{Geschenk des Mondes:} Wenn dein Ablagestapel leer ist, hat diese Gabe für dich keine Auswirkung.
	
	\medskip
	
	\emph{Geschenk des Flusses:} Ziehe die zusätzliche Karte erst, nachdem du deine Kartenhand für den nächsten Zug gezogen hast.
	
	\medskip
	
	\emph{Geschenk des Himmels:} Wenn du weniger als drei Handkarten hast, darfst du alle deine Handkarten ablegen, erhältst dafür aber kein \emph{GOLD}. Hast du drei oder mehr Karten auf der Hand, musst du genau 3 Karten oder gar keine ablegen. Nur wenn du genau 3 Karten ablegst, nimmst du ein \emph{GOLD}.}
\end{tikzpicture}
\hspace{-0.6cm}
\begin{tikzpicture}
	\card
	\cardstrip
	\cardbanner{banner/purple.png}
	\cardtitle{Plagen (1/4)\qquad}
	\cardcontent{Alle 12 \emph{Plagen} sind jeweils nur 1x im Spiel enthalten. Sie werden nur erhalten bzw. empfangen, wenn eine Königreichkarte mit dem Typ UNHEIL dies anweist.
		
	\smallskip
		
	Die Anweisung einer \emph{Plage} wird erst ausgeführt, sobald sie \emph{empfangen} wird. Nach dem Empfang der \emph{Plage} wird diese auf einen separaten \emph{Plagen}-Ablagestapel gelegt. Sie werden niemals Bestandteil des Kartensatzes eines Spielers.
	
	\medskip
	
	\emph{Elend:} Wenn dich diese \emph{Plage} zum dritten (oder vierten, fünften...) Mal in einem Spiel trifft, passiert nichts. Du bleibst bei \emph{DOPPELT ELENDIG}. Pro Spieler ist eine \emph{Zustandskarte} \emph{ELENDIG/DOPPELT ELENDIG} im Spiel enthalten. Jeder Spieler erhält maximal einen solchen \emph{Zustand} – sie werden nicht zwischen den Spielern ausgetauscht.
	
	\begin{itemize}
		\item[\rightarrow] \emph{Zustand} \emph{ELENDIG:} Solange dieser Zustand vor dir liegt, ist dieser – 2 \victorypoint wert.\\
		\item[\rightarrow] \emph{Zustand} \emph{DOPPELT ELENDIG:} Solange dieser Zustand vor dir liegt, ist dieser – 4 \victorypoint wert.
	\end{itemize}
	}
\end{tikzpicture}
\hspace{-0.6cm}
\begin{tikzpicture}
	\card
	\cardstrip
	\cardbanner{banner/purple.png}
	\cardtitle{Plagen (2/4)\qquad}
	\cardcontent{\emph{Furcht:} Du musst eine Aktions- oder Geldkarte deiner Wahl ablegen, wenn du mindestens eine auf der Hand hast. Du zeigst deine Kartenhand nur vor, wenn du keines von beiden auf der Hand hast.
		
	\medskip
		
	\emph{Heuschrecken:} Wenn du etwas anderes als ein \emph{ANWESEN} oder ein \emph{KUPFER} entsorgst, musst du eine billigere Karte desselben Typs (\emph{GELD}, \emph{AKTION}, \emph{ANGRIFF}, \emph{NACHT}, \emph{PUNKTE} etc.) nehmen, sofern es eine gibt. Bei Karten mit mehreren Typen muss es mindestens eine Übereinstimmung im Typ geben.
	
	\smallskip
	
	Wenn du z.B. einen \emph{SCHUSTER} (Typen: \emph{NACHT} \& \emph{DAUER}, Wert: \coin[5] ) entsorgst, kannst du dir eine \emph{GEHEIME HÖHLE} (Typen: \emph{AKTION} \& \emph{DAUER}; Wert: \coin[3] ) nehmen.
	
	\medskip
	
	\emph{Krieg:} Findest du in deinem Nachziehstapel keine Karte, die \coin[3] oder \coin[4] kostet (auch nach dem Mischen des Ablagestapels), entsorge keine Karte und lege alle aufgedeckten Karten auf den Ablagestapel.}
\end{tikzpicture}
\hspace{-0.6cm}
\begin{tikzpicture}
	\card
	\cardstrip
	\cardbanner{banner/purple.png}
	\cardtitle{Plagen (3/4)\qquad}
	\cardcontent{\emph{Neid:} Hast du bereits den \emph{Zustand} \emph{GETÄUSCHT}/\emph{NEIDISCH} vor dir liegen (egal mit welcher Seite nach oben), passiert nichts. Hast du ihn noch nicht vor dir liegen, nimm ihn und lege ihn mit der Seite \emph{NEIDISCH} nach oben vor dir ab. Lege den \emph{Zustand} zu Beginn deiner nächsten Kaufphase zurück und führe den \emph{Zustand} \emph{NEIDISCH} aus. Pro Spieler ist eine Zustandskarte \emph{GETÄUSCHT}/\emph{NEIDISCH} im Spiel enthalten. Jeder Spieler erhält maximal einen solchen \emph{Zustand} – sie werden nicht zwischen den Spielern ausgetauscht.
		
	\begin{itemize}
		\item[\rightarrow] \emph{Zustand Neidisch:} Lege den \emph{Zustand} zu Beginn deiner nächsten Kaufphase zurück. \emph{GOLD} und \emph{SILBER} sind ab diesem Moment bis zum Ende deines Zuges \coin[1] wert. Andere Geldkarten sind nicht betroffen.
	\end{itemize}
	
	\medskip
	
	\emph{Schlechtes Omen:} Normalerweise führt diese \emph{Plage} dazu, dass dein Nachziehstapel nur aus 2 \emph{KUPFER} besteht und der Rest auf dem Ablagestapel liegt. Hast du nur 1 \emph{KUPFER}, liegt diese als einzige Karte auf dem Nachziehstapel. Hast du kein \emph{KUPFER}, ist dein Nachziehstapel leer – zeige deinen Ablagestapel vor, um dies nachzuweisen.}
\end{tikzpicture}
\hspace{-0.6cm}
\begin{tikzpicture}
	\card
	\cardstrip
	\cardbanner{banner/purple.png}
	\cardtitle{Plagen (4/4)\qquad}
	\cardcontent{\emph{Täuschung:} Hast du bereits den \emph{Zustand} \emph{GETÄUSCHT}/\emph{NEIDISCH} vor dir liegen (egal mit welcher Seite nach oben), passiert nichts. Hast du ihn noch nicht vor dir liegen, nimm ihn und lege ihn mit der Seite \emph{GETÄUSCHT} nach oben vor dir ab. Lege den \emph{Zustand} zu Beginn deiner nächsten Kaufphase zurück und führe den \emph{Zustand} \emph{GETÄUSCHT} aus. Pro Spieler ist eine Zustandskarte \emph{GETÄUSCHT}/\emph{NEIDISCH} im Spiel enthalten. Jeder Spieler erhält maximal einen solchen \emph{Zustand} – sie werden nicht zwischen den Spielern ausgetauscht.
		
	\begin{itemize}
		\item[\rightarrow] \emph{Zustand Getäuscht:} Lege den \emph{Zustand} zu Beginn deiner nächsten Kaufphase zurück. Ab diesem Moment bis zum Ende deines Zuges darfst du keine Aktionskarten kaufen.
	\end{itemize}
	}
\end{tikzpicture}
\hspace{-0.6cm}
\begin{tikzpicture}
	\card
	\cardstrip
	\cardbanner{banner/white.png}
	\cardtitle{Zustände\qquad}
	\cardcontent{\tiny {\emph{Elendig:} Nimm den \emph{Zustand} \emph{ELENDIG}, wenn du das erste Mal die \emph{Plage} \emph{ELEND} empfängst. Solange dieser Zustand vor dir liegt, ist dieser – 2 \victorypoint wert.
		
	\medskip
		
	\emph{Doppelt Elendig:} Drehe \emph{ELENDIG} auf \emph{DOPPELT ELENDIG} um, wenn du das zweite Mal die \emph{Plage} \emph{ELEND} empfängst. Solange dieser Zustand vor dir liegt, ist dieser – 4 \victorypoint wert.
	
	\medskip
	
	\emph{Getäuscht:} Nimm den \emph{Zustand} \emph{GETÄUSCHT}, wenn dieser (oder \emph{NEIDISCH}) nicht bereits vor dir liegt, wenn du die \emph{Plage} \emph{TÄUSCHUNG} empfängst. Lege den \emph{Zustand} zu Beginn deiner nächsten Kaufphase zurück. Ab diesem Moment bis zum Ende deines Zuges darfst du keine Aktionskarten kaufen.
	
	\medskip
	
	\emph{Im Wald verirrt:} Nimm den \emph{Zustand} \emph{IM WALD VERIRRT}, wenn dieser nicht bereits vor dir liegt, wenn du die Königreichkarte \emph{NARR} ausspielst. Der \emph{Zustand} bleibt vor dir liegen, bis ein anderer Spieler ihn mit Hilfe des \emph{NARREN} erhält. Solange der \emph{Zustand} vor dir liegt, kannst du dessen Anweisung zu Beginn jedes Zuges anwenden (optional).
	
	\medskip
	
	\emph{Neidisch:} Nimm den \emph{Zustand} \emph{NEIDISCH}, wenn dieser (oder \emph{GETÄUSCHT}) nicht bereits vor dir liegt, wenn du die \emph{Plage} \emph{NEID} empfängst. Lege den \emph{Zustand} zu Beginn deiner nächsten Kaufphase zurück. \emph{GOLD} und \emph{SILBER} sind ab diesem Moment bis zum Ende deines Zuges \coin[1] wert. Andere Geldkarten sind nicht betroffen.\\}
	}
\end{tikzpicture}
\hspace{-0.6cm}
\begin{tikzpicture}
	\card
	\cardstrip
	\cardbanner{banner/white.png}
	\cardicon{icons/coin.png}
	\cardprice{0\textsuperscript{*}}
	\cardtitle{Wunsch}
	\cardcontent{Diese Karte wird nur verwendet, wenn die Königreichkarte \emph{KOBOLD} und/oder \emph{GEHEIME HÖHLE} (\rightarrow ERBSTÜCK \emph{WUNDERLAMPE}) verwendet wird.
		
	\medskip
		
	Diese Karte kann nicht gekauft werden – sie kann nur durch die Anweisung auf der Königreichkarte \emph{KOBOLD} oder dem ERBSTÜCK \emph{WUNDERLAMPE} genommen werden.
	
	\medskip
	
	Du darfst nur dann eine Karte nehmen, wenn du den \emph{WUNSCH} auf seinen Stapel zurückgelegt hast. Tust du das nicht, z.B. weil du ihn mit Hilfe des \emph{THRONSAALS} (aus dem \emph{Basisspiel}) doppelt ausgespielt hast und ihn beim zweiten Mal nicht mehr zurücklegen kannst, darfst du keine Karte nehmen. Nimmst du eine Karte, die normalerweise woanders hingelegt werden würde (z.B. \emph{NOMADENCAMP} aus \emph{Hinterland}), nimm sie trotzdem auf die Hand.}
\end{tikzpicture}
\hspace{-0.6cm}
\begin{tikzpicture}
	\card
	\cardstrip
	\cardbanner{banner/black.png}
	\cardicon{icons/coin.png}
	\cardprice{2\textsuperscript{*}}
	\cardtitle{Fledermaus}
	\cardcontent{Diese Karte kann nicht gekauft werden – sie kann nur durch die Anweisung auf der Königreichkarte \emph{VAMPIRIN} genommen werden.
		
	\medskip
		
	Diese Karte ist eine Nachtkarte und kann nur in der Nachtphase ausgespielt werden. Wenn du diese \emph{FLEDERMAUS} in eine \emph{VAMPIRIN} eintauschst, lege die \emph{FLEDERMAUS} zurück auf ihren Stapel. Ist keine \emph{VAMPIRIN} mehr im Vorrat vorhanden, kannst du die \emph{FLEDERMAUS} nicht eintauschen, du kannst sie aber trotzdem ausspielen und Karten entsorgen.}
\end{tikzpicture}
\hspace{-0.6cm}
\begin{tikzpicture}
	\card
	\cardstrip
	\cardbanner{banner/gold.png}
	\cardtitle{Erbstücke (1/2)\qquad}
	\cardcontent{ERBSTÜCKE werden ausschließlich in der Spielvorbereitung verteilt – und nur, wenn Königreichkarten im Spiel verwendet werden, die ein entsprechendes ERBSTÜCK erfordern. Alle nicht benötigten ERBSTÜCKE kommen in diesem Spiel nicht zum Einsatz. ERBSTÜCKE sind Geldkarten (\emph{WEIDELAND} ist zusätzlich eine Punktekarte) mit einer zusätzlichen Anweisung und ersetzen in der Spielvorbereitung jeweils 1 \emph{KUPFER}:
		
	\medskip
		
	\emph{Wunderlampe:} Dieses ERBSTÜCK wird nur verwendet, wenn die Königreichkarte \emph{GEHEIME HÖHLE} verwendet wird.\\
	Die ausgespielte \emph{WUNDERLAMPE} selbst zählt als eine der 6 Karten, wenn du von ihr genau 1 Karte im Spiel hast. Auch wenn du die \emph{WUNDERLAMPE} entsorgst, erhältst du das \coin[1] für diesen Zug.
	
	\medskip
	
	\emph{Zauberspiegel:} Dieses ERBSTÜCK wird nur verwendet, wenn die Königreichkarte \emph{FRIEDHOF} verwendet wird.\\
	Du darfst diesen \emph{ZAUBERSPIEGEL} nur entsorgen, wenn du dies durch die Anweisung einer anderen Karte tun darfst – der \emph{ZAUBERSPIEGEL} selbst gibt dir dazu nicht das Recht. Solltest du aber eine Möglichkeit haben, diesen \emph{ZAUBERSPIEGEL} zu entsorgen, darfst du eine Aktionskarte aus der Hand ablegen und einen \emph{GEIST} von seinem Stapel nehmen.}
\end{tikzpicture}
\hspace{-0.6cm}
\begin{tikzpicture}
	\card
	\cardstrip
	\cardbanner{banner/gold.png}
	\cardtitle{Erbstücke (2/2)\qquad}
	\cardcontent{\emph{Beutel:} Dieses ERBSTÜCK wird nur verwendet, wenn die Königreichkarte \emph{FÄHRTENSUCHER} verwendet wird.
		
	\medskip
		
	\emph{Weideland:} Dieses ERBSTÜCK wird nur verwendet, wenn die Königreichkarte \emph{SCHÄFERIN} verwendet wird.\\
	Als Geldkarte ausgespielt, ist \emph{WEIDELAND} \coin[1] wert. Zusätzlich bringt sie pro \emph{ANWESEN} im Kartensatz eines Spielers 1 \victorypoint .
	
	\medskip
	
	\emph{Ziege:} Dieses ERBSTÜCK wird nur verwendet, wenn die Königreichkarte \emph{FEE} verwendet wird. Das Entsorgen einer Handkarte ist optional.
	
	\medskip
	
	\emph{Glückstaler:} Dieses ERBSTÜCK wird nur verwendet, wenn die Königreichkarte \emph{NARR} verwendet wird.
	
	\medskip
	
	\emph{Verfluchtes Gold:} Dieses ERBSTÜCK wird nur verwendet, wenn die Königreichkarte \emph{PUKA} verwendet wird.}
\end{tikzpicture}
\hspace{-0.6cm}
\begin{tikzpicture}
	\card
	\cardstrip
	\cardbanner{banner/white.png}
	\cardicon{icons/coin.png}
	\cardprice{3}
	\cardtitle{Zombies}
	\cardcontent{ZOMBIES werden nur im Spiel verwendet und in der Spielvorbereitung in den Müll gelegt, wenn die Königreichkarte \emph{TOTENBESCHWÖRER} im Spiel verwendet wird.
		
	\medskip
		
	\emph{Zombie-Lehrling:} Nur, wenn du eine Karte aus der Hand entsorgst, ziehst du 3 Karten nach und erhältst + 1 Aktion.
	
	\medskip
	
	\emph{Zombie-Maurer:} Du musst, auch wenn du eine Karte entsorgt hast, keine Karte nehmen. Du kannst auch nur eine Karte entsorgen und nichts weiter tun. Du kannst, wenn du eine Karte nimmst, auch eine Karte mit gleichen Kosten oder eine billigere nehmen, auch eine gleiche wie die entsorgte Karte.
	
	\medskip
	
	\emph{Zombie-Spion:} Ziehe eine Karte, bevor du dir die oberste Karte des Nachziehstapels ansiehst.}
\end{tikzpicture}
\hspace{-0.6cm}
\begin{tikzpicture}
	\card
	\cardstrip
	\cardbanner{banner/white.png}
	\cardicon{icons/coin.png}
	\cardprice{0\textsuperscript{*}}
	\cardtitle{Irrlicht}
	\cardcontent{Diese ERSCHEINUNG wird nur verwendet, wenn die Königreichkarte \emph{EXORZISTIN} und/oder eine beliebige Königreichkarte mit dem Typ SEGEN (\rightarrow \emph{GESCHENK DES SUMPFES}) verwendet wird.
		
	\medskip
		
	Diese Karte kann nicht gekauft werden – sie kann nur durch die Anweisung auf der Königreichkarte \emph{EXORZISTIN} oder der \emph{Gabe} \emph{GESCHENK DES SUMPFES} genommen werden.
	
	\medskip
	
	Kostet die aufgedeckte Karte nicht \coin[2] oder weniger, lege sie auf deinen Nachziehstapel zurück.}
\end{tikzpicture}
\hspace{-0.6cm}
\begin{tikzpicture}
	\card
	\cardstrip
	\cardbanner{banner/white.png}
	\cardicon{icons/coin.png}
	\cardprice{2\textsuperscript{*}}
	\cardtitle{Teufelchen}
	\cardcontent{Diese ERSCHEINUNG wird nur verwendet, wenn mindestens eine der Königreichkarten \emph{EXORZISTIN}, \emph{TEUFELSWERKSTATT} und/oder \emph{FOLTERKNECHT} verwendet wird.
		
	\medskip
		
	Diese Karte kann nicht gekauft werden – sie kann nur durch eine Anweisung auf den Königreichkarten \emph{EXORZISTIN}, \emph{TEUFELSWERKSTATT} oder \emph{FOLTERKNECHT} genommen werden.
	
	\medskip
	
	Du darfst eine Aktionskarte aus deiner Hand ausspielen, von der du gerade keine gleiche im Spiel hast. Du kannst eine Karte ausspielen, die du in diesem Zug gespielt, aber bereits entsorgt hast. Dauerkarten aus vergangenen Zügen befinden sich im Spiel und dürfen entsprechend nicht ausgespielt werden. Nicht ausgespielt werden darf ein weiteres \emph{TEUFELCHEN}, da ein solches sich bereits im Spiel befindet.}
\end{tikzpicture}
\hspace{-0.6cm}
\begin{tikzpicture}
	\card
	\cardstrip
	\cardbanner{banner/orangeblack.png}
	\cardicon{icons/coin.png}
	\cardprice{4\textsuperscript{*}}
	\cardtitle{Geist}
	\cardcontent{\tiny{Diese ERSCHEINUNG wird nur verwendet, wenn eine der Königreichkarten \emph{EXORZISTIN} und/oder \emph{FRIEDHOF} (\rightarrow ERBSTÜCK \emph{ZAUBERSPIEGEL}) verwendet wird.
			
	\medskip
			
	Diese Karte kann nicht gekauft werden – sie kann nur durch die Anweisung auf der Königreichkarte \emph{EXORZISTIN} oder dem ERBSTÜCK \emph{ZAUBERSPIEGEL} genommen werden.
	
	\medskip
	
	Diese Karte ist eine Nachtkarte und kann nur in der Nachtphase ausgespielt werden. Wenn dein Nachziehstapel aufgebraucht ist, bevor du eine Aktionskarte aufdeckst, lege die bereits aufgedeckten Karten zur Seite, mische deinen Ablagestapel und lege ihn als neuen Nachziehstapel bereit. Wenn du trotzdem keine Aktionskarte findest, lege alle aufgedeckten Karten ab und es passiert nichts weiter. Lege in diesem Fall den \emph{GEIST} am Ende des Zuges ab (normalerweise erst in der Aufräumphase des nächsten Zuges). Wenn du eine Aktionskarte findest, musst du sie zusammen mit diesem \emph{GEIST} zur Seite legen und zu Beginn deines nächsten Zuges zweimal ausspielen – dies ist nicht optional. Ist die zur Seite gelegte Aktionskarte zusätzlich eine Dauerkarte, bleibt auch der \emph{GEIST} solange im Spiel, bis die Dauerkarte abgelegt wird.
	
	\medskip
	
	Solltest du zu Beginn deines Zuges mehrere Dauerkarten mit \enquote{Zu Beginn des Zuges}-Anweisungen im Spiel haben, entscheidest du, in welcher Reihenfolge du sie abhandelst. Sobald du die Aktionskarte abwickelst, musst du sie hintereinander zweimal ausspielen – du darfst keine andere Anweisung dazwischen abhandeln. Spiele die Aktionskarte aus, führe ihre Anweisungen aus und spiele sie ein zweites Mal aus. Dies verbraucht keine freien oder zusätzlich durch + x Aktionen erhaltenen Aktionen. Sollte sich die Aktionskarte selbst entsorgen, führe ihre Anweisungen trotzdem ein zweites Mal aus, auch wenn sie nicht mehr im Spiel ist.}\\}
\end{tikzpicture}
\hspace{-0.6cm}
\begin{tikzpicture}
	\card
	\cardstrip
	\cardbanner{banner/white.png}
	\cardtitle{\scriptsize{Spielvorbereitung (1/3)}\qquad}
	\cardcontent{Verwendet ihr eine oder mehrere Königreichkarten mit einem gelben ERBSTÜCK-Banner, erhält jeder Spieler ein entsprechendes ERBSTÜCK und dafür ein \emph{KUPFER} weniger. Spielt ihr zum Beispiel mit den Königreichkarten \emph{FEE} und \emph{NARR}, erhält jeder Spieler zu Beginn des Spiels 3 \emph{ANWESEN}, 5 \emph{KUPFER}, 1 \emph{GLÜCKSTALER} und 1 \emph{ZIEGE}.
		
	\medskip
		
	Verwendet ihr eine oder mehrere Königreichkarten mit dem Typ SEGEN, mischt ihr alle \emph{Gaben} und legt sie neben dem Vorrat bereit (\emph{Gaben} sind kein Teil des Vorrats). Legt außerdem den ERSCHEINUNGS-Stapel \emph{IRRLICHT} neben dem Vorrat bereit.
	
	\medskip
	
	Verwendet ihr eine oder mehrere Königreichkarten mit dem Typ UNHEIL, mischt ihr alle \emph{Plagen} und legt sie neben dem Vorrat bereit (\emph{Plagen} sind kein Teil des Vorrats). Legt außerdem die \emph{Zustände} \emph{ELENDIG}/\emph{DOPPELT ELENDIG} sowie \emph{NEIDISCH}/\emph{GETÄUSCHT} neben dem Vorrat bereit.}
\end{tikzpicture}
\hspace{-0.6cm}
\begin{tikzpicture}
	\card
	\cardstrip
	\cardbanner{banner/white.png}
	\cardtitle{\scriptsize{Spielvorbereitung (2/3)}\qquad}
	\cardcontent{Verwendet ihr folgende Königreichkarten, beachtet bitte die entsprechende Vorbereitung:
		
	\medskip
		
	\underline{\emph{Gaben \& Plagen}}\\
	\emph{DRUIDIN:} Legt die 3 obersten \emph{Gaben} aufgedeckt zur Seite – direkt neben den \emph{DRUIDINNEN}-Stapel. Verwendet ihr einen der empfohlenen Königreichkartensätze mit der \emph{DRUIDIN}, legt die entsprechend vorgegebenen \emph{Gaben} aufgedeckt zur Seite.
	
	\medskip
	
	\underline{\emph{Zombies}}\\
	\emph{TOTENBESCHWÖRER:} Legt die 3 ZOMBIES offen in den Müll.
	
	\medskip
	
	\underline{\emph{Zustände}}\\
	\emph{NARR:} Legt den \emph{Zustand} \emph{IM WALD VERIRRT} neben dem Vorrat bereit.}
\end{tikzpicture}
\hspace{-0.6cm}
\begin{tikzpicture}
	\card
	\cardstrip
	\cardbanner{banner/white.png}
	\cardtitle{\scriptsize{Spielvorbereitung (3/3)}\qquad}
	\cardcontent{\underline{\emph{Erscheinungen}}\\
	\emph{EXORZISTIN:} Legt alle 3 ERSCHEINUNGS-Stapel (\emph{GEIST}, \emph{TEUFELCHEN}, \emph{IRRLICHT}) neben dem Vorrat bereit.\\
	\emph{FRIEDHOF:} Legt den \emph{GEIST}-Stapel neben dem Vorrat bereit.\\
	\emph{TEUFELSWERKSTATT} und/oder \emph{FOLTERKNECHT}: Legt den \emph{TEUFELCHEN}-Stapel neben dem Vorrat bereit.
	
	\medskip
	
	\underline{\emph{Wünsche}}\\
	\emph{KOBOLD:} Legt den \emph{WUNSCH}-Stapel neben dem Vorrat bereit.\\
	\emph{GEHEIME HÖHLE:} Legt den \emph{WUNSCH}-Stapel neben dem Vorrat bereit.
	
	\medskip
	
	\underline{\emph{Fledermäuse}}\\
	\emph{VAMPIRIN:} Legt den \emph{FLEDERMAUS}-Stapel neben dem Vorrat bereit.}
\end{tikzpicture}
\hspace{-0.6cm}
\begin{tikzpicture}
	\card
	\cardstrip
	\cardbanner{banner/white.png}
	\cardtitle{\footnotesize{Neue Regeln (1/5)}\qquad}
	\cardcontent{\emph{Neue Spielphase \enquote{Nacht}:} In \emph{Nocturne} gibt es einen neuen Kartentyp – die NACHT-Karten. In Spielen, in denen mindestens 1 Königreichkarte dieses Typs verwendet wird, schließt sich unmittelbar an die Kaufphase (vor der Aufräumphase) die Nachtphase an – in dieser Phase dürfen ausschließlich Karten des Typs NACHT ausgespielt werden. Es darf eine beliebige Anzahl Nachtkarten ausgespielt werden.
		
	\medskip
		
	\emph{Erbstücke:} In \emph{Nocturne} gibt es Königreichkarten, die ein gelbes Banner tragen. Verwendet ihr im Spiel eine oder mehrere dieser Karten, ersetzt jeder Spieler in der Spielvorbereitung ein \emph{KUPFER} durch das oder die entsprechenden ERBSTÜCKE.}
\end{tikzpicture}
\hspace{-0.6cm}
\begin{tikzpicture}
	\card
	\cardstrip
	\cardbanner{banner/white.png}
	\cardtitle{\footnotesize{Neue Regeln (2/5)}\qquad}
	\cardcontent{\emph{Segen \& Gaben\textsuperscript{*}:} In \emph{Nocturne} gibt es Königreichkarten mit dem Typ SEGEN. Verwendet ihr im Spiel eine oder mehrere dieser Karten, werden alle \emph{Gaben} zu Beginn des Spiels gemischt und als verdeckter Stapel neben dem Vorrat bereitgelegt. Sie gehören nicht zum Vorrat.
		
	\medskip
		
	Die Königreichkarten mit dem Typ SEGEN enthalten Anweisungen, die Spielern in irgendeiner Art und Weise \emph{Gaben} einbringen. Die Anweisung \enquote{Empfange eine \emph{Gabe}} bedeutet, dass der Spieler die oberste Karte des \emph{Gaben}-Stapels aufdeckt und die Anweisung darauf befolgt. Empfangene \emph{Gaben} werden mit Ausnahme von \emph{GESCHENK DES FELDES}, \emph{GESCHENK DES WALDES} und \emph{GESCHENK DES FLUSSES} direkt auf einen separaten \emph{Gaben}-Ablagestapel gelegt. Die vorgenannten \emph{Gaben} werden nach Empfang vor dem Spieler bis zu dessen Aufräumphase abgelegt und dann erst auf den Ablagestapel gelegt. Ist der \emph{Gaben}-Stapel leer, wird der \emph{Gaben}-Ablagestapel gemischt und als neuer \emph{Gaben}-Stapel bereitgelegt.}
\end{tikzpicture}
\hspace{-0.6cm}
\begin{tikzpicture}
	\card
	\cardstrip
	\cardbanner{banner/white.png}
	\cardtitle{\footnotesize{Neue Regeln (3/5)}\qquad}
	\cardcontent{\emph{Unheil \& Plagen\textsuperscript{*}:} In \emph{Nocturne} gibt es Königreichkarten mit dem Typ UNHEIL. Verwendet ihr im Spiel eine oder mehrere dieser Karten, werden alle \emph{Plagen} zu Beginn des Spiels gemischt und als verdeckter Stapel neben dem Vorrat bereitgelegt. Sie gehören nicht zum Vorrat.
		
	\medskip
		
	Die Königreichkarten mit dem Typ UNHEIL enthalten Anweisungen, die Spielern in irgendeiner Art und Weise \emph{Plagen} bescheren. Die Anweisung \enquote{Empfange eine \emph{Plage}} bedeutet, dass der Spieler die oberste Karte des \emph{Plagen}-Stapels aufdeckt und die Anweisung darauf befolgt. Die Anweisung \enquote{Alle Mitspieler empfangen die nächste \emph{Plage}} bedeutet, dass die oberste \emph{Plage} aufgedeckt wird und alle Mitspieler die Anweisung derselben Karte (im Uhrzeigersinn) befolgen müssen. Empfangene \emph{Plagen} werden immer direkt auf einen separaten \emph{Plagen}-Ablagestapel gelegt. Sobald alle \emph{Plagen} empfangen wurden, wird der \emph{Plagen}-Ablagestapel gemischt und als neuer \emph{Plagen}-Stapel bereitgelegt.}
\end{tikzpicture}
\hspace{-0.6cm}
\begin{tikzpicture}
	\card
	\cardstrip
	\cardbanner{banner/white.png}
	\cardtitle{\footnotesize{Neue Regeln (4/5)}\qquad}
	\cardcontent{\emph{Zustände\textsuperscript{*}:} In \emph{Nocturne} gibt es drei \emph{Plagen} und eine Königreichkarte, die Spielern einen \emph{Zustand} verschaffen. \emph{Zustände} sind Karten, die vor einem Spieler abgelegt werden und eine zusätzliche Regel beinhalten. Es gibt zwei \emph{Zustände}, die einen einzelnen Zug betreffen und dann zurückgelegt werden (\emph{GETÄUSCHT} und \emph{NEIDISCH}), zwei \emph{Zustände}, die die Punktewertung beeinflussen (\emph{ELENDIG} und \emph{DOPPELT ELENDIG}) und einen \emph{Zustand}, der alle Züge eines Spielers beeinflusst, bis ein anderer Spieler den \emph{Zustand} erhält (\emph{IM WALD VERIRRT}). Die Zustände \emph{GETÄUSCHT}/\emph{NEIDISCH} sowie \emph{ELENDIG}/\emph{DOPPELT ELENDIG} sind jeweils auf einer Karte (Vorder- und Rückseite) – die jeweils gültige Seite liegt oben. Ein \emph{Zustand} ist nur so lange gültig, wie die entsprechende Karte vor einem Spieler liegt.}
\end{tikzpicture}
\hspace{-0.6cm}
\begin{tikzpicture}
	\card
	\cardstrip
	\cardbanner{banner/white.png}
	\cardtitle{\footnotesize{Neue Regeln (5/5)}\qquad}
	\cardcontent{\textsuperscript{*} \emph{Zustände}, \emph{Plagen} und \emph{Gaben} gehören nicht zum Vorrat und sind keine \enquote{Karten} im Sinne des Spiels. Sie werden nicht beachtet, wenn es darum geht \enquote{eine Karte zu nehmen} oder wenn alle \enquote{Karten im Spiel} betrachtet werden. Sie werden ebenso wie die Ereignisse aus \emph{Abenteuer} und die Landmarken aus \emph{Empires} niemals in das Kartendeck eines Spielers integriert.
		
	\medskip
		
	\emph{Die Dauerkarten:} In \emph{Nocturne} gibt es Dauerkarten, die z.B. bereits aus \emph{Seaside} und \emph{Abenteuer} bekannt sind. Die orangefarbenen Dauerkarten beinhalten Anweisungen, die zu einem späteren Zeitpunkt ausgeführt werden. Sie werden normalerweise nicht in der Aufräumphase des Zuges abgelegt, in dem sie ausgespielt wurden, sondern bleiben bis zur Aufräumphase des Zuges, in dem sie letztmals eine Wirkung haben, im Spiel. Wird eine Dauerkarte mehrfach ausgespielt (z.B. durch den \emph{THRONSAAL} aus dem \emph{Basisspiel}), bleibt die verursachende Karte ebenfalls solange im Spiel, bis die Dauerkarte abgelegt wird.
	
	\medskip
	
	Um anzuzeigen, dass eine Dauerkarte in der aktuellen Aufräumphase noch nicht abgelegt wird, wird sie in eine eigene Reihe oberhalb der restlichen ausgespielten Karten gelegt.}
\end{tikzpicture}
\hspace{-0.6cm}
\begin{tikzpicture}
	\card
	\cardstrip
	\cardbanner{banner/white.png}
	\cardtitle{\scriptsize{Empfohlene 10er Sätze\qquad}}
	\cardcontent{\emph{Abenddämmerung:}\\
	Folterknecht, Getreuer Hund, Kloster, Nachtwache, Narr (\rightarrow Im Wald verirrt), Schäferin, Schuster, Seliges Dorf, Sündenpfuhl, Tragischer Held

	\medskip

	\emph{Mitternacht:}\\
	Druidin (\rightarrow Geschenk des Feuers, \rightarrow Geschenk des Sumpfes, \rightarrow Geschenk des Windes), Exorzistin (\rightarrow Geist, Teufelchen, \rightarrow Irrlicht), Geheime Höhle (\rightarrow Wunsch), Kobold, Konklave, Krypta, Plünderer, Puka, Teufelswerkstatt (\rightarrow Teufelchen), Verfluchtes Dorf

	\medskip

	\emph{Nachtschicht} (+ \textit{Basisspiel 2. Edition}): \\
	Druidin (\rightarrow Geschenk der Erde, \rightarrow Geschenk des Feuers, \rightarrow Geschenk des Waldes), Exorzistin (\rightarrow Geist, \rightarrow Teufelchen, \rightarrow Irrlicht), Geisterstadt, Götze, Nachtwache, \textit{Banditin}, \textit{Gärten}, \textit{Mine}, \textit{Schmiede}, \textit{Wilddiebin}

	\medskip

	\emph{Müßiggang} (+ \textit{Basisspiel 2. Edition}): \\
	Konklave, Minnesängerin, Teufelswerkstatt (\rightarrow Teufelchen), Tragischer Held, Verfluchtes Dorf, \textit{Geldverleiher}, \textit{Händlerin}, \textit{Keller}, \textit{Markt}, \textit{Vorbotin}}
\end{tikzpicture}
\hspace{-0.6cm}
\begin{tikzpicture}
	\card
	\cardstrip
	\cardbanner{banner/white.png}
	\cardtitle{\scriptsize{Empfohlene 10er Sätze\qquad}}
	\cardcontent{\emph{Das neue Schwarz} (+ \textit{Seaside}): \\
	Geheime Höhle (\rightarrow Wunsch), Geisterstadt, Plünderer, Schuster, Sündenpfuhl, \textit{Außenposten}, \textit{Hafen}, \textit{Handelsschiff}, \textit{Karawanenwächter}, \textit{Taktiker}
	
	\medskip
	
	\emph{Luftschloss} (+ \textit{Empires} + \underline{Ereignisse und Landmarken-Karten}): \\
	Exorzistin (\rightarrow Geist, \rightarrow Teufelchen, \rightarrow Irrlicht), Friedhof, Narr (\rightarrow Im Wald verirrt), Schäferin, Wechselbalg, \textit{Archiv}, \textit{Katapult/Felsen}, \textit{Ingenieurin}, \textit{Schlösser}, \textit{Tempel}, \underline{Grabmal}
	
	\medskip
	
	\emph{Puka-Possen} (+ \textit{Empires} + \underline{Ereignisse und Landmarken-Karten}): \\
	Attentäter, Fee, Geisterstadt, Getreuer Hund, Puka, \textit{Forum}, \textit{Gärtnerin}, \textit{Opfer}, \textit{Siedler/Emsiges Dorf}, \textit{Wagenrennen}, \underline{Bankett}}
\end{tikzpicture}
\hspace{-0.6cm}
\begin{tikzpicture}
	\card
	\cardstrip
	\cardbanner{banner/white.png}
	\cardtitle{Platzhalter\quad}
\end{tikzpicture}
\hspace{0.6cm}
